\documentclass{article}

\usepackage{amsfonts}
\usepackage{amsmath}
\usepackage{amssymb}
\usepackage{graphicx}
\usepackage{graphicx}
\usepackage{xcolor}

\begin{document}

\newcommand{\R}{\mathbb{R}}

\title{HW4 MATH 670}
\author{Fernando}
\date{\today}
\maketitle

\section*{\#1}
\section*{\#2}
We can write the vector field in local coordinates like this
\[
  V=\sum v_kX_k
\]
then
\[
  V'(t)=\sum v_k'(t)X_k.
\]
On the other hand
\[
  \frac{DV}{dt}=\sum_{i,j,k}\left(\frac{dv_k}{dt}+\frac{d\alpha_i}{dt}v_j\Gamma^k_{ij}\right)X_k.
\]
In this case \(\frac{\partial d\alpha_i}{dt}=0\), so it doesn't matter what \(\Gamma^k_{ij}\) is. We will get:
\[
  \frac{DV}{dt}=\sum\frac{dv_k}{dt}X_k=\sum v_k'(t)X_k=V'(t).
\]
\section*{\#3}

\end{document}
