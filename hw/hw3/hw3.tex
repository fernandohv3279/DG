\documentclass{article}

\usepackage{amsfonts}
\usepackage{amsmath}
\usepackage{amssymb}
\usepackage{graphicx}
\usepackage{graphicx}
\usepackage{xcolor}

\begin{document}

\newcommand{\R}{\mathbb{R}}

\title{HW3 MATH 670}
\author{Fernando}
\date{\today}
\maketitle

\section*{\#1}
\subsection*{(a)}
In this case it is enough to prove that when we take the Lie bracket of two
right invariant vector fields we get a right invariant vector field.

Applying Lemma 3.3.7 with $M=N=G$ and $f=R_g$ immediately gives us
\[
  (dR_g)_h([X,Y](h))=[X,Y](hg),
\]
which is what we wanted.
\subsection*{(b)}
By hypothesis
\[
  (dL_g)(X)=X,
\]
then applying dinv on both sides we get
\begin{align*}
  d\text{inv}(dL_g(X))&=d\text{inv}(X)\\
  d(\text{inv}\circ L_g)(X)&=d\text{inv}(X)\\
  d(R_{g^{-1}}\circ \text{inv})(X)&=d\text{inv}(X)\\
  (dR_{g^{-1}})(d\text{inv}(X))&=d\text{inv}(X),
\end{align*}
similar to theorem 3.3.7.

In order to see that $d\text{inv}(X)_e=-X_e$ I used the following results I found in
``Foundations of differential manifolds and Lie groups" by Frank Warner. There is
probably a simpler way to do this but I don't know.\\
\includegraphics[width=\textwidth]{diagram.png}
\includegraphics[width=\textwidth]{properties.png}

Then according to theorem 3.32 above ($\varphi=$inv) we have
\begin{align*}
  \text{inv}(\exp(X))&=\exp(\text{inv}(X))\\
  (\exp(X))^{-1}&=\exp(\text{inv}(X))\\
  \exp(-X)&=\exp(\text{inv}(X)),
\end{align*}
where we used theorem 3.31.c above for the last line.

By the injectivity of the exponential map we have the result.
\subsection*{(c)}

\section*{\#2}
\section*{\#3}
\subsection*{(a)}
\subsection*{(b)}
\subsection*{(c)}

\end{document}
