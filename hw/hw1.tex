\documentclass{article}
\usepackage{amsmath}
\usepackage{amsfonts}
\usepackage{graphicx}

\begin{document}

\newcommand{\R}{\mathbb{R}}

\title{HW1 MATH 670}
\author{Fernando}
\date{\today}
\maketitle

\section*{\#1}
\subsection*{(a)}
Consider the charts $(\R,\phi_N)$ and
$(\R,\phi_S)$, where $\phi_N$ and $\phi_S$ are
defined in example 1.15 of the notes (inverse
stereographic projection from the north and south poles
respectively).\\
\textbf{Claim:} This collection of charts
works.

\textbf{Proof:}

We have two cases: $\alpha=\beta$ and $\alpha\neq\beta$.
\begin{itemize}
	\item $\alpha=\beta$: This case is probably
		trivial but I will check it anyway.
		Here
		$\phi_\beta^{-1}\circ\phi_\alpha=I:\R^n\to\R^n$,
		then for any point $p$,
		$dI_p:T_p\R^n\to T_p\R^n$ is the
		identity map. Which has determinant
		1.

	\item $\alpha\neq\beta$: WLOG take $\phi_\beta=\phi_N$ and
		$\phi_\alpha=\phi_S$ then (as in page 4 of the notes)
		$(\phi_N^{-1}\circ\phi_S)(\vec{x})=\frac{1}{||\vec{x}||^2}\vec{x}$.
		If we pick the curve $\alpha(t)=p+tv$ we get:
		\begin{align*}
			\left(\frac{p+tv}{||p+tv||^2}\right)'(0)
			&=\left(\frac{v||p+tv||^2-(p+tv)\left(2(p+tv)\cdot v\right)}{||p+tv||^4}\right)(0)\\
			&=\frac{v||p||^2-2p(p\cdot v)}{||p||^4}\\
			&=\frac{v||p||^2-2pp^Tv}{||p||^4}\\
			&=\frac{||p||^2I-2pp^T}{||p||^4}v\\
			&=\frac{1}{||p||^2}\left(I-\frac{2}{||p||^2}pp^T\right)v.
		\end{align*}
		Then using the Weinstein–Aronszajn identity (thanks professor!) we can compute the determinant like this:
		\begin{align*}
\bigg|\frac{1}{||p||^2}\left(I-\frac{2}{||p||^2}pp^T\right)\bigg|
&= \frac{1}{||p||^{2n}}\bigg|\left(I-\frac{2}{||p||^2}pp^T\right)\bigg|\\
&=\frac{2}{||p||^{2n}}\bigg|\left(1-\frac{2}{||p||^2}p^Tp\right)\bigg|\qquad(\text{W-A Identity})\\
&=\frac{-1}{||p||^{2n}}.
		\end{align*}
\end{itemize}
% TODO: Change it to be positive!!


\subsection*{(b)}
\section*{\#2}
\section*{\#3}
\section*{\#4}
\end{document}
