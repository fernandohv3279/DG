\documentclass{article}
\usepackage{amsmath}
\usepackage{amssymb}
\usepackage{amsfonts}
\usepackage{graphicx}
\usepackage{xcolor}

\begin{document}

\newcommand{\R}{\mathbb{R}}

\title{HW2 MATH 670}
\author{Fernando}
\date{\today}
\maketitle

\section*{\#1}
\subsection*{(a)}
\subsection*{(b)}
\subsection*{(c)}
\section*{\#2}
\subsection*{$\implies$}

Extend $\{v_1,\dots,v_k\}$ to a basis for $V$. Let's denote this basis
$\{v_1,\dots,v_n\}$. Proposition 2.4.5 gives us a basis for $\bigwedge^k(V)$.
One of the elements of this basis is exactly $v_1\wedge \cdots\wedge v_k$. In
particular, this element must be different than 0 because it is an element of a
basis.

\subsection*{$\impliedby$}
Let's prove the contrapositive.

If the vectors are l.d. then we can write $v_i=\sum_{j\neq i}a_jv_j$ for some
$i\in\{1,\dots,k\}$. Then
\begin{align*}
	v_1\wedge\cdots\wedge v_i \wedge\cdots\wedge v_k =&
	v_1\wedge\cdots\wedge \left(\sum_{j\neq i}a_jv_j\right) \wedge\cdots\wedge v_k\\
	=& \sum_{j\neq i}a_j \left(v_1\wedge\cdots\wedge v_j \wedge\cdots\wedge v_k\right)\\
	=& 0.
\end{align*}
\section*{\#3}

\subsection*{(a)}
An element of the form $v\wedge w +x\wedge y$ is decomposable iff $v,w,x,y$ are l.d. Let's prove this.
\subsubsection*{$\implies$}
We have that $v\wedge w +x\wedge y=z\wedge k$. Then
\begin{align*}
	0=z\wedge k\wedge z\wedge k&=(v\wedge w +x\wedge y) \wedge (v\wedge w +x\wedge y)\\
	&=x\wedge y \wedge v \wedge w + v \wedge w \wedge x\wedge y\\
	&=2v \wedge w \wedge x\wedge y.
\end{align*}
By problem 2 this meand that $v,w,x,y$ are l.d.
\subsubsection*{$\impliedby$}
W.L.O.G suppose $w=\alpha v+\beta x+\gamma y$. Then
\begin{align*}
	v\wedge w + x\wedge y =&
	v\wedge(\alpha v + \beta x + \gamma y) + x\wedge y\\
	=& v\wedge\beta x + v\wedge\gamma y + x\wedge y\\
	=& x\wedge(-\beta v) + v\wedge\gamma y + x\wedge y\\
	=& x\wedge(-\beta v+y) + v\wedge\gamma y \\
	=& x\wedge(-\beta v+y) + \gamma v\wedge y \\
	=& x\wedge(-\beta v+y) + \gamma v\wedge(-\beta v + y) \\
	=& (x+\gamma v)\wedge(-\beta v+y).
\end{align*}

\subsection*{(b)}
\subsubsection*{$\implies$}
Since $\omega$ is decomposable, write $\omega=x\wedge y$. Then
\[
	\omega\wedge\omega=x\wedge y\wedge x\wedge y=0.
\]
\subsubsection*{$\impliedby$}
we proceed by contrapositive. Suppose that $\omega$ is not decomposable, then
by part a we can write $\omega=v\wedge w+x\wedge y$ with $v,w,x,y$ l.i. then
\begin{align*}
	\omega\wedge\omega&=\left(v\wedge w+x\wedge y\right)\wedge \left(v\wedge w+x\wedge y\right)\\
	&=x\wedge y \wedge v \wedge w + v \wedge w \wedge x\wedge y\\
	&=2v \wedge w \wedge x\wedge y.
\end{align*}
By problem 2 this is not 0 because $v,w,x,y$ are l.i.

\section*{\#4}
\subsection*{(a)}
\subsection*{(b)}
\subsection*{(c)}
\section*{\#5}
\end{document}
