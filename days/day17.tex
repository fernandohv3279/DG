% !TEX root = ../dg.tex

\section{Integration on Submanifolds}
\label{sec:integration on submanifolds}

So far, we've only defined integrating an $n$-form over an $n$-dimensional manifold. However, notice that any open subset of an $n$-dimensional manifold is also an $n$-manifold, so we can actually integrate over arbitrary open subsets.

Indeed, given an $n$-manifold $M$ and any $\omega \in \Omega^n(M)$, we get a (signed) measure $m$ on open subsets of $M$ defined by
\[
	m(A):=\int_A \omega.
\]
When $\omega $ is a volume form (meaning it is nowhere-vanishing), then $m(A)$ will have the same sign for all open $A$, and if the sign is positive then $m$ will satisfy the axioms of a measure on the $\sigma$-algebra of Borel sets on $M$; that is, it is a Borel measure. Moreover, if $M$ is compact then $\int_M \omega$ is finite, so $m$ can be normalized to give a probability measure. Indeed, starting with a volume form is by far the most common way of defining probability measures on (orientable) manifolds.

However, this is not so useful for the integrals mentioned in \Cref{sec:de Rham} that show up in de Rham cohomology. After all, if $N$ is a $k$-dimensional submanifold of the $n$-dimensional manifold $M$  with $k < n$ and $\omega \in \Omega^n(M)$, then $\int_N \omega = 0$.\footnote{As written, this doesn't quite make sense given our definitions, but one way to formalize it would be to realize $N$ as the intersection of a decreasing nested sequence of open sets and then taking the limit of the integral of $\omega$ over those open sets. The measures of these sets will go to zero.}

Rather, for de Rham cohomology purposes, we need a way of defining $\int_N \eta$ where $\eta \in \Omega^k(M)$. Intuitively, we want to ``restrict'' $\eta$ to $N$ here, but trying to write this intuition out directly leads to something quite complicated. But this is a place where all our effort in defining things in such a relatively abstract way pays off by giving us a simple and natural (though seemingly slightly indirect) way of defining our integral:

\begin{definition}\label{def:integration on submanifold}
	Suppose $N$ is a $k$-dimensional submanifold of the manifold $M$ and let $i \from N \hookrightarrow M$ be the inclusion map. If $\eta \in \Omega^k(M)$, define
	\[
		\int_N \eta := \int_N i^\ast \eta.
	\]
\end{definition}

The point here is that $i^\ast \eta \in \Omega^k(N)$, and \Cref{def:integration} applies.

\begin{example}\label{ex:surface integral}
	Let $M = \R^3$ and let $S$ be a surface which is the graph of a function; i.e.,
	\[
		S = \{(u,v,g(u,v)) : u,v \in \R\},
	\]
	for some smooth function $g \from \R^2 \to \R$ (this is not really a restrictive assumption: the implicit function theorem implies that locally every surface looks like this, possibly after permuting coordinates). We can parametrize $S$ by the map $h \from \R^2 \to \R^3$ given by $h(u,v) = (u,v,g(u,v))$.
	
	Suppose $\omega \in \Omega^2(\R^3)$ is compactly supported. We can write
	\[
		\omega = f_1 \, dy\wedge dz + f_2 \, dz \wedge dx + f_3 \, dx \wedge dy.
	\]
	Then
	\[
		\int_S \omega = \int_{\R^2} h^\ast \omega,
	\]
	so we can compute
	\begin{align*}
		h^\ast dx & = d(x \circ h) = du \\
		h^\ast dy & = d(y \circ h) = dv \\
		h^\ast dz & = d(z \circ h) = d(g(u,v)) = \frac{\partial g}{\partial u} du + \frac{\partial g}{\partial v} dv,
	\end{align*}
	and hence
	\begin{align*}
		h^\ast (dx \wedge dy) & = h^\ast dx \wedge h^\ast dy = du \wedge dv \\
		h^\ast (dy \wedge dz) & = h^\ast dy \wedge h^\ast dz = dv \wedge \left(\frac{\partial g}{\partial u} du + \frac{\partial g}{\partial v} dv \right) = -\frac{\partial g}{\partial u} du  \wedge dv \\
		h^\ast (dz \wedge dx) & = h^\ast dz \wedge h^\ast dx = \left( \frac{\partial g}{\partial u} du + \frac{\partial g}{\partial v} dv\right) \wedge du = -\frac{\partial g}{\partial v} du \wedge dv.
	\end{align*}
	Therefore,
	\begin{equation}\label{eq:surface integral formula} 
		\int_S \omega = \int_{\R^2} h^\ast \omega = \int_{\R^2} \left( -\frac{\partial g}{\partial u}f_1 -\frac{\partial g}{\partial v}f_2 + f_3\right) du \wedge dv.
	\end{equation}
	
	Now, I claim this is a standard formula from vector calculus. To see this, let $\vec{F} = (f_1, f_2, f_3)$ be the vector field corresponding to $\omega$ (in the notation from \Cref{sec:vector calculus}, $\omega = \star F^\flat$) and let $\vec{n} = \left( -\frac{\partial g}{\partial u}, -\frac{\partial g}{\partial v}, 1\right)$. Notice that, at $p = (u,v,g(u,v)) \in S$, the tangent space $T_pS$ is spanned by
	\[
		\left. \frac{d}{dt} \right|_{t=0} h(t,0) = \left. \frac{d}{dt} \right|_{t=0} (t,0,g(t,0)) = \left( 1, 0 , \frac{\partial g}{\partial u} \right)
	\]
	and
	\[
		\left. \frac{d}{dt} \right|_{t=0} h(0,t) = \left. \frac{d}{dt} \right|_{t=0} (0,t,g(0,t)) = \left( 0,1 , \frac{\partial g}{\partial v} \right).
	\]
	Since
	\[
		\vec{n} \cdot \left( 1, 0 , \frac{\partial g}{\partial u} \right) = \left( -\frac{\partial g}{\partial u}, -\frac{\partial g}{\partial v}, 1\right) \cdot \left( 1, 0 , \frac{\partial g}{\partial u} \right) = -\frac{\partial g}{\partial u} + \frac{\partial g}{\partial u}  = 0
	\]
	and
	\[
		\vec{n} \cdot \left( 0,1 , \frac{\partial g}{\partial v} \right) = \left( -\frac{\partial g}{\partial u}, -\frac{\partial g}{\partial v}, 1\right) \cdot \left(0,1 , \frac{\partial g}{\partial v} \right) = -\frac{\partial g}{\partial v} + \frac{\partial g}{\partial v}  = 0,
	\]
	the vector $\vec{n}$ is normal to $S$ at all points of $S$. 
	
	Therefore, if $\vec{u} = \frac{\vec{n}}{\|\vec{n}\|}$ and $\dA = \|\vec{n}\| du \wedge dv$ (usually called the area form or area element for $S$), then our formula~\eqref{eq:surface integral formula} can be re-written as
	\[
		\int_S \omega = \int_{\R^2} \left(\vec{F} \cdot \vec{u}\right) \dA,
	\]
	which is the standard formula from vector calculus for computing the surface integral of a vector field.
\end{example}

\begin{example}
	Recall the form $\omega = \frac{x \, dy - y \, dx}{x^2 + y^2} \in \Omega^1(\R^2 -\{0\})$ defined in \Cref{ex:closed not exact} and also considered in \Cref{ex:closed not exact 2}. Let $f \from S^1 \to \R^2 -\{0\}$ defined by $f(\theta) = (\cos \theta , \sin \theta)$ be the standard parametrization of the unit circle. Then
	\[
		\int_{S^1} \omega = \int_{S^1} f^\ast \omega.
	\]
	We saw in \Cref{ex:unit circle pullback dx} that $f^\ast dx = -\sin\theta\, d\theta$ and a similar computation shows that $f^\ast dy = \cos \theta\, d\theta$. Since
	\[
		\omega_{f(\theta_0)} = \frac{\cos(\theta_0) \, dy - \sin(\theta_0) \, dx}{\cos^2 \theta_0 + \sin^2\theta_0} = \cos(\theta_0) \, dy - \sin(\theta_0) \, dx,
	\]
	we see that
	\[
		(f^\ast \omega)_{\theta_0} = (\cos^2\theta_0 + \sin^2\theta_0)\, d\theta = d\theta,
	\]
	so
	\[
		\int_{S^1} \omega = \int_{S^1} f^\ast \omega = \int_{S^1} d\theta = 2\pi.
	\]
\end{example}

\section{Stokes' Theorem} 
\label{sec:stokes_theorem}

At this point we have all the ingredients in place to prove (a version of) de Rham's theorem: for a $k$-dimensional homology class $[C] \in H_k(M;\R)$ represented by a submanifold $C$, and a de Rham cohomology class $[\omega]$ represented by a closed form $\omega \in \Omega^k(M)$, we at least know what
\[
	I([\omega])([C]) := \int_C\omega
\]
means.

However, to see that this is well-defined, we need to know that, if $[\omega_1] = [\omega_2]$, then $\int_C \omega_1 = \int_C \omega_2$, and similarly that if $[C_1] = [C_2]$, then $\int_{C_1}\omega = \int_{C_2}\omega$.

For the first, suppose $\omega_1, \omega_2 \in \Omega^k(M)$ represent the same de Rham cohomology class (i.e., the same element of $\hdr^k(M) = \frac{\ker d}{\im d}$), then $\omega_1 - \omega_2 = d\eta$ for some $\eta \in \Omega^{k-1}(M)$. So we need to show that
\[
	0 = \int_C d\eta = \int_C(\omega_1 - \omega_2) = \int_C \omega_1 - \int_C \omega_2.
\]
Since $C$ is a cycle, $\partial C = \emptyset$, so this will follow from:

\begin{theorem}[Stokes' Theorem for Differential Forms]\label{thm:stokes}
	If $M$ is a compact, oriented $n$-manifold with boundary and $\omega \in \Omega^{n-1}(M)$, then
	\[
		\int_M d\omega = \int_{\partial M} \omega.
	\]
\end{theorem}

As for the second equality we need to show that the pairing $I$ is well, defined, suppose $[C_1] = [C_2]$. This means\footnote{If you haven't seen homology before, you might just need to trust me on this.} that $\partial E = C_2 - C_1$ for some $(k+1)$-chain $E$. But then for any closed $\omega \in \Omega^k(M)$,
\[
	\int_{C_2} \omega - \int_{C_1}\omega = \int_{C_2-C_1} \omega = \int_E d \omega = \int_E 0 = 0
\]
by \Cref{thm:stokes}. 

In other words, Stokes' Theorem is the key to showing that the pairing $I$ is well-defined. Of course, to really properly prove \Cref{thm:de Rham} we would also need to argue that every homology clas can be represented by a submanifold (or at least a finite union of submanifolds), which we're not going to bother with.

To give a complete proof of \Cref{thm:stokes}, we would need to deveop the theory of manifolds with boundary. However, since we won't really be using that theory for anything else, I don't think it's worth the effort. Instead, I'll just give the part of the proof that doesn't interact with the boundary. See Guillemin and Pollack~\cite{guilleminDifferentialTopology2010} for details on manifolds with boundary and a complete proof of Stokes' Theorem.

\begin{proof}[{First part of proof of \Cref{thm:stokes}}]
	Notice, first of all, that by linearity of the integral we can assume $\omega$ has compact support contained in the image of some local coordinate charte $(\phi,U)$, so that $\int_M d\omega = \int_U \phi^\ast d\omega = \int_U d\phi^\ast \omega$. If we assume $\phi(U) \cap \partial M = \emptyset$, then $\int_{\partial M} \omega = 0$, so the goal is to show that $\int_U d\phi^\ast \omega = 0$.
	
	We know that $\phi^\ast \omega \in \Omega^{n-1}(U) \subset \Omega^{n-1}(\R^n)$, so we can write
	\[
		\phi^\ast \omega = \sum_{i=1}^n (-1)^{i-1} f_i dx_1 \wedge \dots \wedge d\check{x}_i \wedge \dots \wedge dx_n
	\]
	for some compactly supported functions $f_i$, where $d\check{x}_i$ means that $dx_i$ is omitted and the signs are chosen so that
	\[
		d\phi^\ast \omega = \left(\sum_{i=1}^n \frac{\partial f_i}{\partial x_i} \right) dx_1 \wedge \dots \wedge dx_n.
	\]
	
	Since the $f_i$ are compactly supported, we can extend their domain of definition to all of $\R^n$ be defining them to be zero outside $U$. Therefore,
	\[
		\int_U d\phi^\ast \omega = \int_U \left(\sum_{i=1}^n \frac{\partial f_i}{\partial x_i} \right) dx_1 \wedge \dots \wedge dx_n = \int_{\R^n} \left(\sum_{i=1}^n \frac{\partial f_i}{\partial x_i} \right) dx_1 \wedge \dots \wedge dx_n = \sum_{i=1}^n\int_{\R^n} \frac{\partial f_i}{\partial x_i}\,  dx_1 \cdots dx_n.
	\]
	
	By Fubini's Theorem, we can interpret each term as an interated integral and do the integration in any order we like:
	\[
		\int_{\R^n} \frac{\partial f_i}{\partial x_i} \, dx_1 \cdots dx_n = \int_{\R^{n-1}} \left(\int_{-\infty}^\infty \frac{\partial f_i}{\partial x_i} \, dx_i\right) dx_1 \cdots d\check{x}_i \cdots dx_n.
	\]
	Now we're thinking of the integrand $\int_{-\infty}^\infty \frac{\partial f_i}{\partial x_i} \, dx_i$ as a function of $(x_1, \dots , \check{x}_i, \dots , x_n)$. Which function? Well, to any specific point $(a_1, \dots, \check{a}_i, \dots a_n)$ it assigns the number $\int_{-\infty}^\infty g'(t)dt$ where $g(t) = f_i(a_1, \dots , t, \dots , a_n)$. But since $f_i$ is compactly supported, we know there exists $T$ big enough that $g(t) = 0$ for all $t \geq T$. 
	
	Then the Fundamental Theorem of Calculus implies that
	\[
		\int_{-\infty}^\infty g'(t) dt = \int_{-T}^T g'(t)dt = g(T) - g(-T) = 0-0=0,
	\]
	so we have
	\[
		\int_M d\omega = \int_U \phi^\ast d\omega = \int_U d\phi^\ast \omega = 0
	\]
	as desired.
\end{proof}

To give the rest of the proof of \Cref{thm:stokes}, one needs to deal with the case $\phi(U) \cap \partial M \neq \emptyset$. Again, see Guillemin and Pollack~\cite{guilleminDifferentialTopology2010} if you're interested.

\begin{example}
	Suppose $S$ is a smooth surface in $\R^3$ with connected boundary $\partial S = \Gamma$, which is parametrized by $\gamma \from S^1 \to \R^3$. Let $\omega \in \Omega^1(\R^3)$. Then \Cref{thm:stokes} tells us that
	\begin{equation}\label{eq:surface stokes}
		\int_S d\omega = \int_\Gamma \omega.
	\end{equation}
	
	Let's translate this into vector calculus language. Let $\vec{F} \in \mathfrak{X}(\R^3)$ be the vector field corresponding to $\omega$. That is, if $\omega = f \, dx + g \, dy + h\, dz$, then $\vec{F} = f\,\frac{\partial}{\partial x} + g\, \frac{\partial}{\partial y} + h \, \frac{\partial}{\partial z}$. 
	
	Following \Cref{ex:line integral},
	\[
		\int_\Gamma \omega = \int_{S^1} \gamma^\ast \omega = \int_{S^1} \vec{F} \cdot \gamma'(t)\, dt
	\]
	which we often write as $\oint_\Gamma \vec{F} \cdot d\Gamma$.
	
	To translate the left-hand side of \eqref{eq:surface stokes}, recall that we saw in \Cref{sec:vector calculus} that the vector field corresponding to $d \omega$ is $\nabla \times \vec{F}$. Then, following \Cref{ex:surface integral} we have
	\[
		\int_S d\omega = \int_{\R^2} \left((\nabla \times \vec{F}) \cdot u\right) dA,
	\]
	which is what we mean in vector calculus by the notation $\iint_S (\nabla \times\vec{F}) \cdot dS$, where $dS$ is the area element on $S$. 
	
	In other words, translating the Stokes' theorem result \eqref{eq:surface stokes} into vector calculus language recovers the vector calculus Stokes' Theorem (or Curl Theorem)
	\[
		\iint_S (\nabla \times\vec{F}) \cdot dS = \oint_\Gamma \vec{F} \cdot d\Gamma.
	\]
\end{example}

\begin{exercise}
	Do the same sort of translation for the Divergence Theorem.
\end{exercise}