% !TEX root = ../dg.tex

\section{De Rham Cohomology}
\label{sec:de Rham}

Recall that we have the cochain complex
	\begin{center}
	\begin{tikzcd}
		\Omega^0(M) \arrow[r,"d_0"] & \Omega^1(M) \arrow[r,"d_1"] &  \dots \arrow[r,"d_{n-2}"] & \Omega^{n-1}(M) \arrow[r,"d_{n-1}"] & \Omega^n(M) 
	\end{tikzcd}
	\end{center}
where the subscript $k$ indicates that $d_k$ is the restriction of $d$ to $\Omega^k(M)$.

\begin{definition}\label{def:de Rham cohomology}
	The \emph{$k$th de Rham cohomology group} of $M$ is
	\[
		\hdr^k(M) := \frac{\ker d_k}{\im d_{k-1}}.
	\]
\end{definition}

\begin{example}\label{ex:closed not exact 2}
	We saw in \Cref{ex:closed not exact} that the 1-form $\omega = \frac{x\, dy - y\, dx}{x^2 + y^2} \in \Omega^1(\R^2 -\{0\})$ is \emph{closed} (in the kernel of $d$), but not \emph{exact} (in the image of $d$), so it represents a non-trivial element of $\hdr^1(\R^2 - \{0\})$.
\end{example}

De Rham cohomology is a homotopy invariant of manifolds, meaning that homotopy equivalent manifolds have identical de Rham cohomology groups. 

If $\omega \in \Omega^k(M)$ and $\eta \in \Omega^\ell(M)$ are closed (i.e., $d \omega = 0$ and $d\eta = 0$), meaning that they represent de Rham cohomology classes in $\hdr^k(M)$ and  $\hdr^\ell(M)$, respectively, then
\[
	d(\omega \wedge \eta) = d\omega \wedge \eta + (-1)^k \omega \wedge d\eta = 0 \wedge \eta + (-1)^k \omega \wedge 0 = 0,
\]
so $d(\omega \wedge \eta)$ is also closed, and hence represents a class in $\hdr^{k+\ell}(M)$. This means that
\[
	\hdr^\ast(M) := \bigoplus_k \hdr^k(M)
\]
forms a (graded) ring, the \emph{de Rham cohomology ring}, with the operation induced by $\wedge$.

\begin{theorem}[de Rham theorem]\label{thm:de Rham}
	There exists a vector space isomorphism
	\[
		\hdr^k(M) \to H_k(M;\R)^\ast \cong H^k(M;\R),
	\]
	where $H_k(M;\R)$ and $H^k(M;\R)$ are the singular (or simplicial) homology and cohomology groups.
\end{theorem}

In fact, even more is true: the de Rham cohomology ring is isomorphic to the singular cohomology ring with real coefficients.

In \Cref{ex:closed not exact 2}, $H^1(\R^2 -\{0\};\R) \cong H^1(S^1;\R) \cong \R$, so the 1-form $\omega = \frac{x\, dy - y\, dx}{x^2 + y^2} \in \Omega^1(\R^2 -\{0\})$ is a generator of $\hdr^1(\R^2-\{0\})$.

We won't give all the details of the proof of de Rham's theorem, but the basic idea is that a closed form defines a linear functional on homology classes by integration. More precisely, the mapping $I \from \hdr^k(M) \to H_k(M;\R)^\ast$ is defined as follows: if $[\omega] \in \hdr^k(M)$ is represented by the closed form $\omega \in \Omega^k(M)$, then $I([\omega]) \in H_k(M;\R)^\ast$ is given by
\[
	I([\omega])([C]) := \int_C \omega
\]
for any $[C] \in H_k(M;\R)$ which is represented by the cycle $C \subset M$. It's (hopefully!) clear that $I([\omega])$ is a linear functional (provided the integral here is anything sensible), but probably less obvious that $I$ is well-defined and an isomorphism. Of course, to even make sense of the definition of $I$, we need to talk about integration.

\section{Integration on Manifolds}
\label{sec:integration_on_manifolds}

As suggested in \Cref{sec:informal def of forms}, the point of defining differential forms as alternating, multilinear gadgets is because they transform in the right way. 

\subsection{Digression on Differentials, Pullbacks, and Dual Maps} 
\label{sub:digression_on_differentials_pullbacks_and_dual_maps}

First, a quick refresher on \emph{dual maps} or \emph{algebraic adjoints}:

Given vector spaces $U$ and $V$ and a linear map $h \from U \to V$, there is an associated dual map $h^\ast \from V^\ast \to U^\ast$ defined as follows: if $\alpha \in V^\ast$, then $h^\ast(\alpha)$ is supposed to be an element of $U^\ast$; that is, a linear functional on $U$, meaning that we can specify it by specifying what it does to an arbitrary element $u \in U$:
\[
	h^\ast(\alpha)(u) := \alpha(h(u)).
\]
In other words, $h^\ast \alpha = \alpha \circ h$. It's easy to check that $h^\ast$ is linear since $h$ is.

Moreover, when $U$ and $V$ are finite-dimensional with chosen bases $u_1, \dots , u_m$ for $U$ and $v_1, \dots , v_n$ for $V$, then the matrix for $h^\ast$ with respect to the dual bases $u_1^\ast, \dots , u_m^\ast$ for $U^\ast$ and $v_1^\ast , \dots , v_n^\ast$ for $V^\ast$ is equal to the transpose (or conjugate transpose, for complex vector spaces) of the matrix for $h$ with respect to the given bases for $U$ and $V$. This explains the ``algebraic adjoint'' terminology, since the transpose (or conjugate transpose) is the usual adjoint, at least when the chosen bases are orthonormal. (And I claim it is occasionally useful to think about transpose in these terms, even though in applications we often think about the transpose as just a map $V \to U$.)

Now, consider a smooth map $g \from M \to N$ and let $p \in M$. Then the differential $dg_p \from T_pM \to T_{g(p)}N$ must have an associated dual map $(dg_p)^\ast \from (T_{g(p)}N)^\ast \to (T_pM)^\ast$.

First of all, notice that $\left(T_{g(p)}N\right)^\ast = \ext{1}\left(\left(T_{g(p)}N\right)^\ast\right)$ and $\left(T_pM\right)^\ast = \ext{1}\left(\left(T_pM\right)^\ast\right)$, so the pullback (at $p$)
\[
	g_p^\ast \from \ext{1}\left(\left(T_{g(p)}N\right)^\ast\right)\to \ext{1}\left(\left(T_pM\right)^\ast\right)
\]
has the same domain and range as $(dg_p)^\ast$.

Moreover, by definition of the dual map, for a linear functional $\alpha \in (T_{g(p)}N)^\ast$ and a tangent vector $v \in T_pM$,
\[
	(dg_p)^\ast (\alpha)(v) = \alpha(dg_p(v)).
\]
But this is exactly how we defined the pullback, where we now think of $\alpha = \omega_{g(p)}$, the value of the 1-form $\omega$ at $g(p)$.

In other words, $g^\ast$ as defined in \Cref{sec:pullbacks} is the same as the dual map $(dg)^\ast$ associated to the differential $dg$.

\subsection{How Differential Forms Transform} 
\label{sub:how_differential_forms_transform}

The goal is to see that transformations of differential forms under changes of coordinates are just like transformations of integrands under changes of coordinates.

As in \Cref{sec:informal def of forms}, let $g \from B \to A$ be a diffeomorphism of open subsets of $\R^n$ and let $f \from A \to \R$ be a smooth function. Define
\[
	\omega = f dx_1 \wedge \dots \wedge dx_n \in \Omega^n(A),
\]
which pulls back to
\[
	g^\ast \omega = (f \circ g) g^\ast (dx_1) \wedge \dots \wedge g^\ast(dx_n) = (f \circ g) (dg)^\ast (dx_1) \wedge \dots \wedge (dg)^\ast (dx_n),
\]
where here $(dg)^\ast$ means the dual of the differential $dg$. By Homework 2, Problem 1(c), the above is equal to 
\[
	(f \circ g) (\det (dg)^\ast) dy_1 \wedge \dots \wedge dy_n,
\]
where we recall that $y_1, \dots , y_n$ were just the names of the coordinates on $B$. Since determinants are invariant under taking transposes, $\det(dg)^\ast = \det dg = \det Jg$, so this parallels the usual vector calculus change of variables formula, at least when $f$ is smooth.

In other words, differential forms transform in the right way to serve as integrands.

\begin{example}
	Let $M = (0, +\infty) \times (-\pi, \pi)$, $N = \R^2$, and the map $g\from M \to N$ given by
	\[
		g(r, \theta) := (r \cos \theta, r \sin \theta).
	\]
	In other words, this is the change of variables map between polar coordinates $(r, \theta)$ and Cartesian coordinates $(x,y)$ on the plane.
	
	Now, consider the standard area form $\omega = dx \wedge dy$ on $N$ and we want to work out $g^\ast \omega$ from the definition. 
	
	By definition of the differential (\Cref{def:differential}), we have that, for $p = (\theta_0, r_0) \in M$, 
	\begin{multline*}
		dg_p\left(\frac{\partial}{\partial \theta} \right) = \left. \frac{d}{dt} \right|_{t=0} g(\theta_0+t,r_0) = \left. \frac{d}{dt} \right|_{t=0} r_0(\cos(\theta_0+t),\sin(\theta_0+t)) \\
		= r_0(-\sin\theta_0,\cos\theta_0) = r_0\left(-\sin\theta_0 \frac{\partial}{\partial x} + \cos \theta_0 \frac{\partial}{\partial y}\right)
	\end{multline*}
	where I move back and forth between tuple representations and local coordinate basis representations. Similarly,
	\begin{multline*}
		dg_p\left(\frac{\partial}{\partial r} \right) = \left. \frac{d}{dt} \right|_{t=0} g(\theta_0,r_0+t) = \left. \frac{d}{dt} \right|_{t=0} (r_0+t)(\cos\theta_0,\sin\theta_0) \\
		= (\cos\theta_0,\sin\theta_0) = \cos\theta_0 \frac{\partial}{\partial x} + \sin \theta_0 \frac{\partial}{\partial y}.
	\end{multline*}
	
	Therefore,
	\begin{align*}
		(g^\ast \omega)_p\left(a \frac{\partial}{\partial \theta} + b \frac{\partial}{\partial r}, c \frac{\partial}{\partial \theta} + d \frac{\partial}{\partial r}\right) & = \omega_{g(p)} \left(dg_p\left(a \frac{\partial}{\partial \theta} + b \frac{\partial}{\partial r}\right), dg_p\left(c \frac{\partial}{\partial \theta} + d \frac{\partial}{\partial r}\right)\right) \\
		& = dx \wedge dy \left((b \cos\theta_0 - a r_0\sin\theta_0) \frac{\partial}{\partial x} + (b\sin \theta_0 +ar_0\cos \theta_0) \frac{\partial}{\partial y}, \right. \\
		& \qquad \qquad \qquad \left. (d \cos\theta_0 - c r_0\sin\theta_0) \frac{\partial}{\partial x} + (d\sin \theta_0 +cr_0\cos \theta_0) \frac{\partial}{\partial y}\right) \\
		& = (b \cos\theta_0 - a r_0\sin\theta_0)(d\sin \theta_0 +cr_0\cos \theta_0) \\
		& \qquad \qquad - (b\sin \theta_0 +ar_0\cos \theta_0)(d \cos\theta_0 - c r_0\sin\theta_0) \\
		& = r(bc-ad).
	\end{align*}
	On the other hand,
	\[
		dr \wedge d\theta \left(a \frac{\partial}{\partial \theta} + b \frac{\partial}{\partial r}, c \frac{\partial}{\partial \theta} + d \frac{\partial}{\partial r}\right) = bc-ad,
	\]
	so we can conclude that $g^\ast \omega = r\, dr \wedge d\theta$.
	
	Since the composition of the constant function 1 (on $N$) with $g$ is just the constant function 1 (on $M$), the above discussion implies that
	\[
		g^\ast \omega = g^\ast(dx \wedge dy) = (\det Jg) dr \wedge d\theta,
	\]
	where $\det Jg$ is the Jacobian determinant
	\[
		\begin{vmatrix} \frac{\partial g_1}{\partial r} & \frac{\partial g_1}{\partial \theta} \\ \frac{\partial g_2}{\partial r} & \frac{\partial g_2}{\partial \theta} \end{vmatrix} = \begin{vmatrix}\cos\theta & -r \sin \theta  \\ \sin \theta  & r \cos \theta\end{vmatrix} = r \cos^2\theta+r\sin^2\theta  = r.
	\]
	So the above discussion of how differential forms transform matches up to the standard vector calculus fact that the Cartesian area element $dx\, dy$ transforms to the polar area element $r\, dr \, d\theta$.
\end{example}

\subsection{Definition of the Integral}
\label{sub:definition_of_the_integral}

First, we define integration of forms on open sets in $\R^n$: if $A \subset \R^n$ is open, then any $\omega \in \Omega^n(A)$ can be written as
\[
	\omega = f\, dx_1 \wedge \dots \wedge dx_n,
\]
where $f$ is a smooth function on $A$. Define
\[
	\int_A \omega = \int_A f \, dx_1 \cdots dx_n,
\]
where the integral on the right is just the usual Riemann integral of the (smooth) function $f$ over $A$.

By the previous discussion, if $g \from B \to A$ is an orientation-preserving diffeomorphism, then $\det Jg > 0$, so
\begin{multline*}
	\int_A \omega = \int_A f \, dx_1 \cdots dx_n = \int_B(f \circ g) |\det Jg| \, dy_1 \cdots dy_n = \int_B (f \circ g) (\det Jg) \, dy_1 \cdots dy_n \\
	 = \int_B(f \circ g) (\det Jg) dy_1 \wedge \dots \wedge dy_n = \int_B g^\ast \omega.
\end{multline*}
In other words, our definition is actually coordinate-independent, as you would hope.

If $g$ is orientation-reversing, then we get
\[
	\int_A \omega = -\int_B g^\ast \omega.
\]

Since we have no standard coordinates on an arbitrary manifold, these transformations are essential to defining integrals on manifolds.

Suppose next that we have an $n$-dimensional manifold $M$ and the support of $\omega \in \Omega^n(M)$ is contained in a single coordinate chart $(U,\phi)$ so that $\phi \from U \to \phi(U)$ is orientation-preserving. Then $\phi^\ast \omega \in \Omega^n(U)$, which we know how to integrate, so define
\[
	\int_M\omega = \int_U \phi^\ast \omega.
\]

\begin{proposition}\label{prop:local integral well-defined}
	The above is well-defined.
\end{proposition}

\begin{proof}
	We need to check that, if the support of $\omega$ is also contained in some other coordinate chart $(V, \psi)$ so that $\psi \from V \to \psi(V)$ is orientation-preserving, then we get the same answer whether we compute $\int_M \omega = \int_U \phi^\ast \omega$ or $\int_M \omega = \int_V \psi^\ast \omega$.
	
	We can move between $U$ and $V$ by way of the map $\psi^{-1} \circ \phi \from U \to V$, which is orientation-preserving, so
	\[
		\int_V \psi^\ast \omega = \int_U(\psi^{-1} \circ \phi)^\ast \psi^\ast \omega = \int_U \phi^\ast (\psi^{-1})^\ast \psi^\ast \omega = \int_U \phi^\ast (\psi \circ \psi^{-1})\ast \omega = \int_U \phi^\ast \omega ,
	\]
	where I'm using the previously unstated fact that, for maps $f \from A \to B$ and $g \from B \to C$, $(g \circ f)^\ast = f^\ast g^\ast$, which follows from the chain rule.
\end{proof}

So we've now succeeded in defining the integral of a form contained entirely in a single local coordinate chart. But what about arbitrary forms?

First of all, to avoid worrying about convergence issues, let's restrict to \emph{compactly supported} forms: that is, those forms whose support (the set of points on which they are non-vanishing) is contained in some compact subset of $M$. Of course, if $M$ is itself compact, this is no restriction at all.

So now suppose we have a compactly supported form $\omega \in \Omega^n(M)$. To define $\int_M \omega$, we will use a partition of unity argument.

\begin{definition}\label{def:partition of unity}
	Let $T$ be a topological space and let $\{W_\alpha\}_{\alpha \in \mathcal{I}}$ be an open cover of $T$, where $\mathcal{I}$ us some indexing set. A \emph{partition of unity} subordinate to $\{W_\alpha\}_{\alpha \in \mathcal{I}}$ is a collection $\{\rho_\alpha\}_{\alpha \in \mathcal{I}}$ of continuous functions $\rho_\alpha \from T \to [0,1]$ so that $\operatorname{supp}(\rho_\alpha) \subset W_\alpha$ (i.e., $\rho_\alpha(x) = 0$ for all $x \notin W_\alpha$) and, for all $x \in T$,
	\begin{itemize}
		\item there exists a neighborhood of $x$ on which all but finitely many of the $\rho_\alpha$ are identically zero and
		\item $\sum_{\alpha \in \mathcal{I}} \rho_\alpha(x) = 1$ (note that the previous condition ensures this sum is finite and hence converges).
	\end{itemize}
\end{definition}

These always exist in reasonable topological spaces:

\begin{theorem}\label{thm:existence of partitions of unity}
	Every open cover on a paracompact Hausdorff space has a subordinate partition of unity.
\end{theorem}

Since manifolds are second countable and Hausdorff, they are paracompact, so this implies (after modifying the proof to make the $\rho_\alpha$ smooth):

\begin{corollary}\label{cor:existence of partitions of unity on manifolds}
	Every manifold has a partition of unity subordinate to its coordinate atlas so that the functions are smooth.
\end{corollary}

Now, finally, we can define integration of compactly-supported forms on manifolds:

\begin{definition}\label{def:integration}
	Let $M$ be an $n$-dimensional manifold and let $\omega \in \Omega^n(M)$ be compactly supported. Let $\{\rho_\alpha\}$ be a partition of unity subordinate to the (maximal) atlas $\{(U_\alpha, \phi_\alpha)\}$ on $M$ and define
	\[
		\int_M \omega = \sum_{\alpha} \int_M \rho_\alpha \omega.
	\]
\end{definition}

\begin{remark}
	\begin{enumerate}
		\item $\rho_\alpha \omega$ is only non-vanishing inside the support of $\rho_\alpha$, which is contained in some coordinate neighborhood $\phi_\alpha(U_\alpha)$, so the integrals inside the sum are well-defined.
		\item The local finiteness of the partition of unity guarantees only finitely many terms are nonzero on the compact support of $\omega$, so the sum converges.
		\item If $\omega$ is already contained in some coordinate chart $(\phi, U)$, then this definition agrees with the previous one: since $\sum_\alpha \rho_\alpha(p) = 1$ for each $p \in M$, we have that $\sum_\alpha \rho_\alpha \omega = \omega$, so using the previous definition
		\[
			\int_M \omega = \int_U \phi^\ast \omega = \int_U \phi^\ast(\sum_\alpha \rho_\alpha \omega) = \sum_\alpha \int_U \phi^\ast(\rho_\alpha \omega) = \sum_\alpha \int_M \rho_\alpha \omega
		\]
		by linearity of pullback and of integration on $\R^n$.
		\item If $\{\sigma_\alpha\}$ were a different subordinate partition of unity, then the previous observation implies that, or each $\alpha$,
		\[
			\int_M \rho_\alpha \omega = \sum_\beta \int_M \sigma_\beta \rho_\alpha \omega
		\]
		and similarly for each $\beta$
		\[
			\int_M \sigma_\beta \omega = \sum_\alpha \int_M \rho_\alpha \sigma_\beta \omega,
		\]
		so
		\[
			\sum_\alpha \int_M \rho_\alpha \omega = \sum_\alpha \sum_\beta \int_M \sigma_\beta \rho_\alpha \omega = \sum_\beta \sum_\alpha \int_M \rho_\alpha \sigma_\beta \omega = \sum_\beta \int_M \sigma_\beta \omega,
		\]
		which means we get the same value for $\int_M \omega$ no matter which partition of unity we use.
		\item Integration is a linear operator: for $a \in \R$ and $\omega, \eta \in \Omega^n(M)$, $\int_M (\omega + a \eta) = \int_M \omega + a \int_M \eta$.
		\item If $\omega \in \Omega^n(M)$ and $g \from N \to M$ is an orientation-preserving diffeomorphism, then
		\[
			\int_M \omega = \int_N g^\ast \omega.
		\]
	\end{enumerate}
\end{remark}

\begin{example}\label{ex:integrating along a curve}
	Let $\gamma \from [a,b] \to M$ be a smooth curve with $\gamma(a) = p$ and $\gamma(b) = q$. Assume $f \in C^\infty(M)$ and $\omega = df \in \Omega^1(M)$. Then I claim that
	\[
		\int_{[a,b]} \gamma^\ast \omega = f(q) - f(p).
	\]
	This is just a computation:
	\begin{multline*}
		\int_{[a,b]} \gamma^\ast \omega = \int_{[a,b]}\gamma^\ast(df) = \int_{[a,b]} d (\gamma^\ast f) = \int_{[a,b]} d(f \circ \gamma) = \int_{[a,b]}(f \circ \gamma)'(t)dt \\
		= (f \circ \gamma)(b) - (f\circ \gamma)(a) = f(q)-f(p)
	\end{multline*}
	where the second equality follows from \Cref{prop:pullbacks commute with exterior derivative}, the third from \Cref{ex:pullbacks of functions are compositions}, and the fifth from the fundamental theorem of calculus.
\end{example}

\begin{example}\label{ex:line integral}
	Suppose $\gamma \from S^1 \to M$ is smooth and $\omega \in \Omega^1(M)$. Then we define the \emph{line integral} of $\omega$ around $\gamma$ to be
	\[
		\oint_\gamma \omega := \int_{S^1} \gamma^\ast \omega.
	\]
	Now suppose $M = \R^n$, so that $\omega = a_1 \, dx_1 + \dots + a_n \, dx_n$. Letting $\left\{\frac{\partial}{\partial \theta}\right\}$ be the local coordinate basis for the tangent space $T_t S^1$ at a point $t \in S^1$ and writing $\gamma(t) = (\gamma_1(t), \dots , \gamma_n(t))$, we can compute
	\[
		(\gamma^\ast dx_i)_t \left(b \frac{\partial}{\partial \theta} \right) = (dx_i)_{\gamma(t)} \left(d\gamma_t\left(b \frac{\partial}{\partial \theta}\right)\right) = b (dx_i)_{\gamma(t)} \left(d\gamma_t\left(\frac{\partial}{\partial \theta}\right)\right) = b (dx_i)_{\gamma(t)} \left(\frac{\partial}{\partial \theta}(\gamma)(t)\right) 
	\]
	using \Cref{lem:vector fields and differentials}. But then
	\[
		\frac{\partial}{\partial \theta}(\gamma)(t) = \gamma_1'(t) \frac{\partial}{\partial x_1} + \dots + \gamma_n'(t) \frac{\partial}{\partial x_n},
	\]
	so we see that 
	\[
		(\gamma^\ast dx_i)_t \left(b \frac{\partial}{\partial \theta} \right) = b \gamma_i'(t)
	\]
	or, in other words, $(\gamma^\ast dx_i)_t = \gamma_i'(\theta)d\theta$. Therefore,
	\[
		(\gamma^\ast \omega)_t = (\gamma^\ast a_1)_t (\gamma^\ast dx_1)_t + \dots + (\gamma^\ast a_n)_t (\gamma^\ast dx_n)_t = \left[a_1(\gamma(t)) \gamma_1'(t) + \dots + a_n(\gamma(t))\gamma_n'(t)\right] d\theta.
	\]
	Notice that the coefficient is just $A(\gamma(t))\cdot \gamma'(t)$, where the vector field $A = (a_1, \dots , a_n)$ is dual to $\omega$. Of course, this is the sort of expression you usually see in the Multivariable Calculus definition of a line integral.
\end{example}

Putting \Cref{ex:integrating along a curve,ex:line integral} together, notice that if $\omega = df \in \Omega^1(M)$, then $\oint_\gamma \omega = 0$ for all closed curves $\gamma$ in $M$.

\begin{proposition}\label{prop:integrate to 0 => exact}
	The converse is true. That is, any $\omega \in \Omega^1(M)$ with the property that $\oint_\gamma \omega = 0$ for all closed curves $\gamma$ in $M$ is exact.
\end{proposition}

\begin{exercise}
	Prove this.
\end{exercise}