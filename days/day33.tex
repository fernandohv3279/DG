% !TEX root = ../dg.tex

\section{The Curvature Tensor}
\label{sec:curvature tensor}

Given the name, you might expect that some of the key ideas in Riemannian geometry are due to Bernhard Riemann. Indeed, he explained a way of thinking about what eventually came to be called Riemannian manifolds and curvature in his \emph{Habilitationsschrift} from 1854~\cite{riemannHypothesesWhichLie2016}.

His idea for how to define curvature in a manifold was something like the following. Let $p \in M$ and consider $\sigma \subset T_pM$, a 2-dimensional subspace. Each line through the origin in $\sigma$ exponentiates out to give a geodesic in $M$, and by taking all such geodesics we get a surface $S$ in $M$ containing $p$ and tangent to $\sigma$ at $p$. The Riemannian metric on $M$ induces a Riemannian metric on $S$.

Gauss had showed in 1827~\cite{gaussGeneralInvestigationsCurved2005} that the Gauss curvature of a surface depends only on its First Fundamental Form (a.k.a., Riemannian metric); Riemann called this the curvature at $p$, denoted $K(p,\sigma)$. These days we call this curvature the \emph{sectional curvature} at $p$ with respect to $\sigma$.

That said, Riemann did not give any way to compute this curvature and it took quite a long time to get a rigorous, modern definition. Unfortunately, Riemann's intuition, which hopefully makes sense, is not really evident in the modern definition, which seems very abstract. 

That said, I'm going to give you the modern definitions and try to connect it back to the geometric intuition as much as I can.

\begin{definition}\label{def:curvature tensor}
	The \emph{(Riemann) curvature tensor} $R$ on a Riemannian manifold $M$ is a map $R \from \mathfrak{X}(M) \times \mathfrak{X}(M) \times \mathfrak{X}(M) \to \mathfrak{X}(M)$ given by
	\begin{equation}\label{eq:curvature tensor}
		R(X,Y)Z = \nabla_X \nabla_Y Z - \nabla_Y \nabla_X Z - \nabla_{[X,Y]}Z,
	\end{equation}
	where $\nabla$ is the Levi-Civita connection associated with the Riemannian metric.
\end{definition}

\begin{remark}
	Some books, including do Carmo~\cite{docarmoRiemannianGeometry1992}, define the curvature tensor to be
	\[
		\nabla_Y\nabla_X Z - \nabla_X\nabla_YZ + \nabla_{[X,Y]}Z;
	\]
	notice that this is $-R(X,Y)Z$ as we've defined it.
\end{remark}

\begin{remark}
	We can see immediately that $R(Y,X)Z = -R(X,Y)Z$, so $R$ is skew-symmetric in the first two factors.
\end{remark}

\begin{remark}
	Given $X,Y \in \mathfrak{X}(M)$, it is often convenient to think of $R(X,Y) \from \mathfrak{X}(M) \to \mathfrak{X}(M)$, sometimes called the \emph{curvature transformation}. Note that
	\[
		R(X,Y) = [\nabla_X,\nabla_Y] - \nabla_{[X,Y]}.
	\]
	If we tried to define a similar curvature transformation for the Lie derivative we would have
	\[
		[L_X,L_Y] - L_{[X,Y]} = 0
	\]
	since
	\begin{multline*}
		L_XL_YZ - L_YL_XZ - L_{[X,Y]}Z  = [X,[Y,Z]] - [Y,[X,Z]] - [[X,Y],Z] \\
		= - [[Y,Z],X] - [[Z,X],Y] - [[X,Y],Z] = 0
	\end{multline*}
	by the Jacobi identity \Cref{prop:Lie bracket}\ref{it:Lie bracket Jacobi identity}.
\end{remark}

\begin{remark}
	In local coordinates, $\left[\frac{\partial}{\partial x_i},\frac{\partial}{\partial x_j} \right] = 0$, so
	\[
		R\left(\frac{\partial}{\partial x_i},\frac{\partial}{\partial x_j}\right)\frac{\partial}{\partial x_k} = \left(\nabla_{\partial/\partial x_i} \nabla_{\partial/\partial x_j} - \nabla_{\partial/\partial x_j} \nabla_{\partial/\partial x_i}\right)\frac{\partial}{\partial x_k},
	\]
	so the curvature tensor is measuring the extent to which iterated covariant differentiation is non-commutative.
\end{remark}

\begin{example}
	In $\R^n$ with the Euclidean metric, $R \equiv 0$. To see this, write $Z = (z_1, \dots , z_n)$ and recall from \Cref{ex:euclidean connection} that covariant differentiation works componentwise:
	\begin{equation}\label{eq:componentwise covariant}
		\nabla_XZ = \sum_k X(z_k)\frac{\partial}{\partial x_k}.
	\end{equation}
	Hence,
	\[
		\nabla_X\nabla_YZ = \sum_k X(Y(z_k))\frac{\partial}{\partial x_k}
	\]
	and similarly for $\nabla_Y\nabla_XZ$, so we get
	\begin{align*}
		R(X,Y)Z & = \nabla_X\nabla_YZ - \nabla_Y\nabla_XZ - \nabla_{[X,Y]}Z \\
		& = \sum_k \left[X(Y(z_k)) - Y(X(z_k)) - [X,Y](z_k)\right]\frac{\partial}{\partial x_k} \\
		& = 0,
	\end{align*}
	so Euclidean space is reassuringly as un-curved as it is possible to be, and suggests that this notion of curvature is some sort of measure of how much the metric on $M$ deviates from being Euclidean.
\end{example}

\begin{example}\label{ex:curvature tensor on S^3}
	Recall our vector fields $X, Y, Z$ on $S^3$ defined in \Cref{sec:S^3 example} as
	\[
		X(x) = xi, \quad Y(x) = xj, \quad Z(x) = xk
	\]
	at each $x \in S^3$.\footnote{I'm switching from $p$ to $x$ as the name of the point in $S^3$ because I'm going to think of points in $S^3$ as points in $\R^4$ and write them as $x = (x_1, x_2,x_3,x_4)$, where the $x_i$ are coordinate functions. This will make it more natural to write things like $\frac{\partial}{\partial x_i}(x_i) = 1$ rather than calling the coordinate functions $p_i$ and trying to remember that $\frac{\partial}{\partial x_i}(p_i) = 1$.} We showed in \Cref{sec:S^3 example} that
	\begin{equation}\label{eq:Lie brackets on S3}
		[X,Y] = 2Z, \quad [Y,Z] = 2X, \quad [Z,X] = 2Y.
	\end{equation}
	
	$S^3$ inherits its standard metric from the Euclidean metric on $\R^4$, and the corresponding Levi-Civita connection $\nabla$ on $S^3$ is just the restriction to $S^3$ (via orthogonal projection) of the Euclidean connection on $\R^4$, which I'll denote as $\nabla^{\R^4}$. In other words, for $U,V \in \mathfrak{X}(S^3)$, $(\nabla_UV)(p)$ is the orthogonal projection of $(\nabla_U^{\R^4}V)(p)$ to $T_xS^3 = x^\bot$.
	
	If I write out $X$ in Euclidean coordinates, I get
	\[
		X(x) = xi = (x_1 + x_2i + x_3j + x_4k)i = -x_2 + x_1i + x_4j - x_3k = -x_2 \frac{\partial}{\partial x_1} + x_1 \frac{\partial}{\partial x_2} + x_4 \frac{\partial}{\partial x_3} - x_3 \frac{\partial}{\partial x_4},
	\]
	and similarly
	\begin{align*}
		Y(x) & = -x_3\frac{\partial}{\partial x_1} -x_4 \frac{\partial}{\partial x_2}+ x_1\frac{\partial}{\partial x_3} + x_2\frac{\partial}{\partial x_4} \\
		Z(x) & = -x_4\frac{\partial}{\partial x_1} +x_3\frac{\partial}{\partial x_2} - x_2\frac{\partial}{\partial x_3} + x_1\frac{\partial}{\partial x_4}.
	\end{align*}
	
	So then, again using~\eqref{eq:componentwise covariant}, we have that
	\begin{align*}
		\nabla_X^{\R^4}Y & = \nabla_X^{\R^4}\left( -x_3\frac{\partial}{\partial x_1} -x_4 \frac{\partial}{\partial x_2}+ x_1\frac{\partial}{\partial x_3} + x_2\frac{\partial}{\partial x_4}\right) \\
		& =  -X(x_3)\frac{\partial}{\partial x_1} -X(x_4) \frac{\partial}{\partial x_2}+ X(x_1)\frac{\partial}{\partial x_3} + X(x_2)\frac{\partial}{\partial x_4} \\
		& = -x_4\frac{\partial}{\partial x_1} +x_3 \frac{\partial}{\partial x_2} - x_2\frac{\partial}{\partial x_3} + x_1\frac{\partial}{\partial x_4}\\
		& = Z,
	\end{align*}
	which is already tangent to $S^3$, so orthogonal projection just gives $Z$ again, and we conclude that $\nabla_XY = Z$.
	
	By similar reasoning, we can compute
	\begin{align}\label{eq:covariant derivatives on S3}
		\nabla_XX & = 0 & \nabla_XY & = Z  & \nabla_XZ & = -Y \nonumber\\
		\nabla_YX & = -Z & \nabla_YY & = 0 &  \nabla_YZ & = X \\
		\nabla_ZX & = Y & \nabla_ZY & = -X & \nabla_ZZ & = 0 \nonumber
	\end{align}
	
	Therefore, we can use~\eqref{eq:Lie brackets on S3} and~\eqref{eq:covariant derivatives on S3} to see that
	\begin{align*}
		R(X,Y)X & = \nabla_X\nabla_YX - \nabla_Y\nabla_XX - \nabla_{[X,Y]}X \\
		& = -\nabla_XZ - \nabla_Y 0 -\nabla_{2Z}X \\
		& = Y-2Y \\
		& = -Y,
	\end{align*}
	that
	\begin{align} \label{eq:RXYY on S3}
		R(X,Y)Y & = \nabla_X\nabla_YY - \nabla_Y\nabla_XY - \nabla_{[X,Y]}Y \nonumber \\
		& = \nabla_X0 - \nabla_Y Z -\nabla_{2Z}Y \\
		& = X+2X \nonumber\\
		& = X, \nonumber
	\end{align}
	and that
	\begin{align*}
		R(X,Y)Z & = \nabla_X\nabla_YZ - \nabla_Y\nabla_XZ - \nabla_{[X,Y]}Z \\
		& = \nabla_XX + \nabla_Y Y -\nabla_{2Z}Z \\
		& = 0+0-0 \\
		& = 0.
	\end{align*}
	Therefore, for each $x \in S^3$ the transformation $R(X,Y) \from T_x S^3 \to T_xS^3$ can be interpreted as the composition of orthogonal projection to the 2-plane spanned by $X$ and $Y$ and a clockwise rotation of this plane by $90^\circ$.
\end{example}

\begin{exercise}\label{ex:curvature transformation on S^3}
	Work out similar interpretations of $R(X,Z)$ and $R(Y,Z)$ on $S^3$.
\end{exercise}

Despite having called it the \emph{curvature tensor}, I have not yet showed that $R$ is a tensor field. To be a proper tensor field, the value of $R(X,Y)Z$ at a point would need to depend \emph{only} on the values of $X,Y,Z$ at the point; given all the derivatives involved, you might expect that it would depend on the values of $X,Y,Z$ in a neighborhood of the point. 

\begin{proposition}\label{prop:multilinearity of R}
	The curvature tensor $R(X,Y)Z$ on a Riemannian manifold $M$ is multilinear in $X,Y,Z$ over the algebra $C^\infty(M)$ of smooth functions on $M$.
\end{proposition}

Of course, \eqref{eq:curvature tensor} shows that $R(X,Y)Z$ is multilinear over $\R$.

\begin{exercise}\label{ex:R depends only on a point}
	Show that multilinearity over $C^\infty(M)$ implies that the value of $R(X,Y)Z$ at $p \in M$ depends only on the values of $X,Y,Z$ at $p$, and not on their values in any neighborhood of $p$.
\end{exercise}

\begin{proof}[Proof of \Cref{prop:multilinearity of R}]
	Suppose $f \in C^\infty(M)$. Then, repeatedly using the parts of \Cref{def:affine connection},
	\begin{align*}
		R(fX,Y)Z & = \nabla_{fX}\nabla_Y Z - \nabla_Y \nabla_{fX}Z - \nabla_{[fX,Y]}Z \\
		& = f\nabla_X\nabla_YZ - \nabla_Y(f\nabla_X Z) - \nabla_{f[X,Y] - Y(f)X}Z \\
		& = f\nabla_X\nabla_YZ - Y(f) \nabla_XZ - f\nabla_Y\nabla_XZ - f\nabla_{[X,Y]}Z + Y(f) \nabla_XZ \\
		& = f(\nabla_X\nabla_YZ- \nabla_Y\nabla_XZ - \nabla_{[X,Y]}Z) \\
		& = f R(X,Y)Z.
	\end{align*}
	Hence, $R(X,Y)Z$ is multilinear in $X$ over $C^\infty(M)$. Since $R(X,Y)Z = -R(Y,X)Z$, this shows that it is also multilinear in $Y$ over $C^\infty(M)$.
	
	Finally,
	\begin{align*}
		R(X,Y)(fZ) & = \nabla_X \nabla_Y fZ - \nabla_Y \nabla_X fZ - \nabla_{[X,Y]}fZ \\
		& = \nabla_X(Y(f)Z + f \nabla_YZ) - \nabla_Y(X(f)Z + f \nabla_XZ) - [X,Y](f)Z - f \nabla_{[X,Y]}Z \\
		& = X(Y(f))Z + Y(f) \nabla_XZ +  X(f) \nabla_Y Z + f \nabla_X \nabla_Y Z \\
		& \qquad \qquad - Y(X(f))Z - X(f) \nabla_YZ - Y(f) \nabla_XZ - f \nabla_Y\nabla_XZ \\
		& \qquad \qquad - X(Y(f))Z + Y(X(f))Z - f \nabla_{[X,Y]}Z \\
		& = f(\nabla_X \nabla_Y Z - \nabla_Y \nabla_X Z - \nabla_{[X,Y]}Z) \\
		& = f R(X,Y)Z,
	\end{align*}
	showing the linearity of $R(X,Y)Z$ in $Z$ over $C^\infty(M)$ and completing the proof.
\end{proof}

\begin{proposition}[Bianchi Identity]\label{prop:Bianchi identity}
	For $M$ a Riemannian manifold and $X,Y,Z \in \mathfrak{X}(M)$,
	\[
		R(X,Y)Z + R(Y,Z)X + R(Z,X)Y = 0.
	\]
\end{proposition}

\begin{proof}
	Since $\nabla$ is symmetric, we know that $\nabla_AB - \nabla_B A = [A,B]$. Therefore, 
	\[
		\nabla_YZ = [Y,Z] + \nabla_ZY
	\]
	and
	\[
		\nabla_{[X,Y]}Z = [[X,Y],Z] + \nabla_Z[X,Y].
	\]
	Therefore,
	\begin{align}\label{eq:R Bianchi 1}
		R(X,Y)Z & = \nabla_X \nabla_Y Z - \nabla_Y \nabla_X Z - \nabla_{[X,Y]}Z \nonumber\\
		& = \nabla_X([Y,Z] + \nabla_ZY) - \nabla_Y \nabla_XZ - [[X,Y],Z] - \nabla_Z[X,Y],
	\end{align}
	and similarly
	\begin{align}
		R(Y,Z)X & = \nabla_Y([Z,X] + \nabla_XZ) - \nabla_Z \nabla_YX - [[Y,Z],X] - \nabla_X[Y,Z] \label{eq:R Bianchi 2} \\
		R(Z,X)Y & = \nabla_Z([X,Y] + \nabla_YX) - \nabla_X \nabla_ZY - [[Z,X],Y] - \nabla_Y[Z,X]. \label{eq:R Bianchi 3}
	\end{align}
	If we add \eqref{eq:R Bianchi 1}, \eqref{eq:R Bianchi 2}, and \eqref{eq:R Bianchi 3}, all the connection terms cancel out and we're left with
	\[
		R(X,Y)Z + R(Y,Z)X + R(Z,X)Y = -([[X,Y],Z] + [[Y,Z],X] + [[Z,X],Y]) = 0
	\]
	by the Jacobi identity \Cref{prop:Lie bracket}\ref{it:Lie bracket Jacobi identity}.
\end{proof}

We've defined $R$ as a map $\mathfrak{X}(M) \times \mathfrak{X}(M) \times \mathfrak{X}(M) \to \mathfrak{X}(M)$; that is, $R(X,Y)Z \in \mathfrak{X}(M)$. We can interpret this as a $(1,3)$-tensor field as follows: at each $p \in M$ we get a map $(T_pM)^\ast \times T_pM \times T_pM \times T_pM \to \R$ given by
\[
	(\tau,X,Y,Z) = \tau(R(X,Y)Z).
\]
(Note that this being well-defined depends essentially on \Cref{ex:R depends only on a point}.)

Of course, we've defined everything in the presence of a Riemannian metric $g$, and we can always convert vectors into linear functionals with inner products: $T \leftrightarrow g(\cdot, T)$ for any $T \in \mathfrak{X}(M)$, and so we can convert $R$ into a $(0,4)$-tensor field—that is, a $C^\infty(M)$-multilinear map $\mathfrak{X}(M) \times \mathfrak{X}(M) \times \mathfrak{X}(M) \times \mathfrak{X}(M) \to \R$ by mapping
\[
	(X,Y,Z,T) \mapsto g(R(X,Y)Z,T).
\]

\begin{proposition}[Symmetries of the Curvature Tensor]\label{prop:symmetries of curvature tensor}
	For $X,Y,Z,T \in \mathfrak{X}(M)$,
	\begin{enumerate}
		\item \label{it:curvature symmetries 4}(Bianchi Identity) $g(R(X,Y)Z,T) + g(R(Y,Z)X,T) + g(R(Z,X)Y,T) = 0$.
		\item \label{it:curvature symmetries 1}(Skew-Symmetry) $g(R(X,Y)Z,T) = -g(R(Y,X)Z,T)$
		\item \label{it:curvature symmetries 2}(Skew-Symmetry) $g(R(X,Y)Z,T) = -g(R(X,Y)T,Z)$
		\item \label{it:curvature symmetries 3}(Interchange Symmetry) $g(R(X,Y)Z,T) = g(R(Z,T)X,Y)$
	\end{enumerate}
\end{proposition}

\begin{proof}
	\ref{it:curvature symmetries 4} follows immediately from \Cref{prop:Bianchi identity}.
	
	\ref{it:curvature symmetries 1} follows from the fact that $R(X,Y) = -R(Y,X)$, which we saw was a direct consequence of the definition.
	
	For \ref{it:curvature symmetries 2}, observe that
	\[
		(XY - YX - [X,Y])(g(Z,T) )= 0
	\]
	since $XY-YX = [X,Y]$. Now, if we repeatedly use \Cref{cor:compatibility equation} we see that
	\begin{align*}
		XY(g(Z,T)) & = X(g(\nabla_Y Z, T) + g(Z,\nabla_YT)) = g(\nabla_X \nabla_Y Z,T) + g(\nabla_YZ,\nabla_XT) + g(\nabla_XZ, \nabla_YT) + g(Z, \nabla_X\nabla_YT) \\
		YX(g(Z,T)) & = Y(g(\nabla_XZ,T) + g(Z,\nabla_XT)) = g(\nabla_Y\nabla_XZ,T) + g(\nabla_XZ,\nabla_YT) + g(\nabla_YZ,\nabla_XT) + g(Z,\nabla_Y\nabla_XT) \\
		{[X,Y](g(Z,T))} & = g(\nabla_{[X,Y]}Z,T) + g(Z,\nabla_{[X,Y]}T).
	\end{align*}
	Subtracting the second and third lines from the first yields
	\begin{align*}
		0 & = (XY - YX - [X,Y])(g(Z,T)) \\
		& = g(\nabla_X\nabla_YZ - \nabla_Y\nabla_X Z - \nabla_{[X,Y]}Z,T) + g(Z,\nabla_X\nabla_YT - \nabla_Y\nabla_X T- \nabla_{[X,Y]}T) \\
		& = g(R(X,Y)Z,T) + g(R(X,Y)T,Z),
	\end{align*}
	which implies \ref{it:curvature symmetries 2}.
	
	For \ref{it:curvature symmetries 3}, note that the Bianchi identity implies that
	\begin{align*}
		g(R(X,Y)Z,T) + g(R(Y,Z)X,T) + g(R(Z,X)Y,T) & = 0 \\
		g(R(Y,Z)T,X) + g(R(Z,T)Y,X) + g(R(T,Y)Z,X) & = 0 \\
		g(R(Z,T)X,Y) + g(R(T,X)Z,Y) + g(R(X,Z)T,Y) & = 0 \\
		g(R(T,X)Y,Z) + g(R(X,Y)T,Z) + g(R(Y,T)X,Z) & = 0 .
	\end{align*}
	Adding the first two lines and subtracting the third and fourth and using \ref{it:curvature symmetries 1} and \ref{it:curvature symmetries 2} to cancel four pairs of terms and simplify the remaining terms yields
	\[
		2g(R(X,Y)Z,T) - 2g(R(Z,T)X,Y) = 0,
	\]
	which implies the result.
\end{proof}

In local coordinates, if we write $X_i = \frac{\partial}{\partial x_i}$, then there exist smooth functions $R_{ijk}^s$ so that
\[
	R(X_j,X_k)X_i = \sum_s R_{ijk}^s X_s.
\]
The functions $R_{ijk}^s$ are usually called the components of $R$ in the given local coordinates. If we have vector fields $U, V, W$ with local coordinate expressions
\[
	U = \sum_j u^j X_j, \quad V = \sum_k v^k X_k, \quad W = \sum_i w^iX_i
\]
(note that now I am being more careful about whether indices are superscripted or subscripted), then
\[
	R(U,V)W = \sum_{i,j,k,s} R_{ijk}^s u^j v^k w^i X_s.
\]
(In Einstein notation, this gets written as $R(U,V)W = R_{ijk}^s u^j v^k w^i X_s$, where summation always happens over repeated indices, and repeated indices should appear in up/down pairs.)

\begin{exercise}
	Show that
	\[
		R_{ijk}^s = \frac{\partial \Gamma_{ik}^s}{\partial x_j} - \frac{\partial \Gamma_{jk}^s}{\partial x_i} + \sum_m(\Gamma_{ik}^m \Gamma_{jm}^s - \Gamma_{jk}^m \Gamma_{im}^s).
	\]
\end{exercise}

When we lower an index and think of $R$ as a $(0,4)$-tensor, then we get components
\[
	R_{ijks} := g(R(X_k,X_s)X_i,X_j) = \left\langle \sum_{m}R_{iks}^m X_m, X_j \right\rangle = \sum_m R_{iks}^m g_{mj}.
\]

Then \Cref{prop:symmetries of curvature tensor} implies the following symmetries for the $R_{ijks}$:

\begin{corollary}\label{cor:symmetries of curvature components}
	\begin{enumerate}
		\item $R_{ijks} + R_{iksj} + R_{isjk} = 0$
		\item $R_{ijks} = -R_{jiks}$
		\item $R_{ijks} = -R_{ijsk}$
		\item $R_{ijks} = R_{ksij}$.
	\end{enumerate}
\end{corollary}

In particular, recall that we proved the interchange symmetry \ref{it:curvature symmetries 3} from the other three, and the first three turn out to be a complete list of symmetries of the curvature tensor, so $R$ is completely specified in local coordinates by the choice of $\frac{n^2(n-1)^2}{12}$ independent functions $R_{ijks}$.