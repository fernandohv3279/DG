% \documentclass{article}
\documentclass{amsart}

\usepackage{amsfonts}
\usepackage{amsmath}
\usepackage{biblatex}
\usepackage{blindtext}
\usepackage{mathtools}
\usepackage{xcolor}

\addbibresource{project.bib}

\begin{document}

\newcommand{\R}{\mathbb{R}}
\newcommand\tbf[1]{\textbf{#1}}
\newcommand\myworries[1]{\textcolor{red}{\tbf{#1}}}
% \renewcommand\myworries[1]{}

\title{Basic differential geometry in cloaking}
\author{Fernando}
\date{\today}
\maketitle

\section{Introduction}
Electromagnetic cloaking is a technique used to make an object invisible or undetectable to electromagnetic waves (like visible light, radar, or microwaves). It works by guiding electromagnetic waves around the object, so they flow as if the object isn’t there.

This is usually done using metamaterials—artificial materials engineered to have specific properties not found in nature. These materials can bend and shape electromagnetic waves in unusual ways, creating a “cloak” that hides the object from detection.

In simple terms: instead of blocking or reflecting light, cloaking materials redirect it around the object, making it appear as if the waves traveled through empty space.

\section{Maxewell's equations}

Maxwell's equations are a set of four fundamental laws that describe how electric and magnetic fields interact and propagate.

(1) Gauss's law for electricity:
\[\nabla \cdot \mathbf{E} = \frac{\rho}{\varepsilon_0},\]
which states that electric charges produce electric fields;

(2) Gauss's law for magnetism,
\[\nabla \cdot \mathbf{B} = 0,\]
indicating that there are no magnetic monopoles;

(3) Faraday's law of induction,
\[\nabla \times \mathbf{E} = -\frac{\partial \mathbf{B}}{\partial t},\]
showing how a changing magnetic field induces an electric field;

and (4) Ampère's law with Maxwell's addition,
\[\nabla \times \mathbf{B} = \mu_0 \mathbf{J} + \mu_0 \varepsilon_0 \frac{\partial \mathbf{E}}{\partial t},\]
which relates magnetic fields to electric currents and changing electric fields.
Together, these equations form the foundation of classical electromagnetism and explain how electromagnetic waves, such as light, propagate through space.

% According to \cite{thompson12}

\section{Theory}
This section was taken from \cite{ward96}, specifically sections 1 and 2.
\myworries{\\Reorganize this section??}
\subsection{Summary of section 1 from Pendry's article}
We start with Maxewell's equations:\myworries{ Is this in Cartesian coordinates? Should we start with cylindrical?}
\begin{align*}
  \nabla \times \tbf{E} &= -\mu \mu_0 \partial \tbf{H} /\partial t\\
  \nabla \times \tbf{H} &= +\varepsilon \varepsilon_0 \partial \tbf{E} /\partial t,
\end{align*}
and then consider a transformation
\[
q_1(x,y,z), \quad q_2(x,y,z), \quad q_3(x,y,z).
\]

\tbf{Question:} What form does the new set of equations take when
expressed in terms of $q_1,q_2,q_3$?

\tbf{Answer:} In the new system of coordinates, Maxwell's equations become
\begin{align*}
  \nabla_q \times \widehat{\tbf{E}} &= -\mu_0\hat{\mu} \partial \widehat{\tbf{H}}/\partial t\\
  \nabla_q \times \widehat{\tbf{H}} &= +\varepsilon_0 \hat{\varepsilon} \partial \widehat{\tbf{E}}/\partial t,
\end{align*}
where $\hat{\mu}$ and $\hat{\varepsilon}$ are in general tensors and $\widehat{\tbf{E}},\widehat{\tbf{H}}$ are the transformed versions of \tbf{E,H}
In other words, the form of Maxewell's equations is preserved by a coordinate transformation, it just changes the definition of $\varepsilon,\mu$.
\subsection{Summary of section 2 from Pendry's article}
We define three unit vectors $\tbf{u}_1,\tbf{u}_2,\tbf{u}_3,$ to point along the generalized $q_1,q_2,q_3$ axes.
We also define
\begin{equation}\label{matrixQ}
  Q_{ij}=\frac{\partial x}{\partial q_i}\frac{\partial x}{\partial q_j}+\frac{\partial y}{\partial q_i}\frac{\partial y}{\partial q_j}+\frac{\partial z}{\partial q_i}\frac{\partial z}{\partial q_j},
\end{equation}
For convenience we will denote $Q_i=\sqrt{Q_{ii}}$.

Another definition we need is
\begin{equation}\label{matrixg}
  g^{-1}=[u_i\cdot u_j]=\begin{bmatrix}
    \tbf{u}_1\cdot \tbf{u}_1 & \tbf{u}_1\cdot \tbf{u}_2 & \tbf{u}_1\cdot \tbf{u}_3 \\
    \tbf{u}_2\cdot \tbf{u}_1 & \tbf{u}_2\cdot \tbf{u}_2 & \tbf{u}_2\cdot \tbf{u}_3 \\
    \tbf{u}_3\cdot \tbf{u}_1 & \tbf{u}_3\cdot \tbf{u}_2 & \tbf{u}_3\cdot \tbf{u}_3
  \end{bmatrix}.
\end{equation}
Now we are ready to define $\hat{\mu},\hat{\varepsilon}$. According to \cite{ward96} they are
\begin{align}
  \hat{\mu}^{ij}=\mu g^{ij}|\tbf{u}_1\cdot(\tbf{u}_2\times \tbf{u}_3)|Q_1Q_2Q_3(Q_iQ_j)^{-1}\\
  \hat{\varepsilon}^{ij}=\varepsilon g^{ij}|\tbf{u}_1\cdot(\tbf{u}_2\times \tbf{u}_3)|Q_1Q_2Q_3(Q_iQ_j)^{-1} \nonumber
\end{align}

\section{Geometric transformation 1}
The transformation we want is the following
\[\begin{cases}
  q_1(r,\theta,z)=a+r(b-a)/b\\
  q_2(r,\theta,z)=\theta\\
  q_3(r,\theta,z)=z\\
\end{cases}\]
Notice that it is writen in terms of cylindrical coordinates
(from cylindrical to cylindrical), but we can move from cylindrical
to cartesian with the usual change of variables:
\begin{equation}
  \begin{cases}\label{cylindrical}
  x=r\cos\theta\\
  y=r\sin\theta\\
  z=z\\
\end{cases}
\end{equation}

Let's apply the formulas from the previous section to the transformation:
\[\begin{cases}
  q_1(x,y,z)=a+r(b-a)/b\\
  q_2(x,y,z)=\theta\\
q_3(x,y,z)=z\\
\end{cases}.\]
Here I kept the transformation in terms of $r,\theta,z$ in order to have cleaner looking formulas.
We can write $r,\theta,z$ in terms of $x,y,z$ if we wanted.

Now we neeq $x,y,z$ in terms of $q_1,q_2,q_3$, for this we note that:
\[\begin{cases}
r=\frac{q_1-a}{b-a}b=\frac{b}{b-a}q_1-\frac{ba}{b-a}\\
\theta=q_2\\
z=q_3\\
\end{cases}.\]
Again this is in terms of $r,\theta,z$ for simplicity, but combining this with (\ref{cylindrical})
we get $x,y,z$ in terms of $q_1,q_2,q_3$.

Now we use (\ref{matrixQ}) to compute $Q_{ij}$.
\begin{align*}
Q_{11}
=&\frac{\partial x}{\partial q_1}\frac{\partial x}{\partial q_1}
+\frac{\partial y}{\partial q_1}\frac{\partial y}{\partial q_1}
+\frac{\partial z}{\partial q_1}\frac{\partial z}{\partial q_1}\\
=&\left(\frac{\partial x}{\partial q_1}\right)^2+\left(\frac{\partial y}{\partial q_1}\right)^2+0\\
=&\left(\frac{\partial (r\cos\theta)}{\partial q_1}\right)^2+\left(\frac{\partial (r\sin\theta)}{\partial q_1}\right)^2\\
=&\left(\cos\theta\frac{\partial r}{\partial q_1}\right)^2+\left(\sin\theta\frac{\partial r}{\partial q_1}\right)^2\\
=&\left(\frac{\partial r}{\partial q_1}\right)^2\\
=&\left(\frac{b}{b-a}\right)^2
\end{align*}

\begin{align*}
Q_{22}
=&\frac{\partial x}{\partial q_2}\frac{\partial x}{\partial q_2}
+\frac{\partial y}{\partial q_2}\frac{\partial y}{\partial q_2}
+\frac{\partial z}{\partial q_2}\frac{\partial z}{\partial q_2}\\
=&\left(\frac{\partial x}{\partial q_2}\right)^2+\left(\frac{\partial y}{\partial q_2}\right)^2+0\\
=&\left(r\frac{\partial \cos\theta}{\partial q_2}\right)^2+\left(r\frac{\partial \sin\theta}{\partial q_2}\right)^2\\
=&\left(-r\sin\theta\frac{\partial \theta}{\partial q_2}\right)^2+\left(r\cos\theta\frac{\partial \theta}{\partial q_2}\right)^2\\
=&r^2\left(\frac{\partial \theta}{\partial q_2}\right)^2\\
=&r^2
\end{align*}

\begin{align*}
Q_{33}
=&\frac{\partial r}{\partial q_3}\frac{\partial r}{\partial q_3}
+\frac{\partial \theta}{\partial q_3}\frac{\partial \theta}{\partial q_3}
+\frac{\partial z}{\partial q_3}\frac{\partial z}{\partial q_3}\\
=&0+0+1\\
=&1
\end{align*}

For $Q_{ij}$ where $i\neq j$ we have $Q_{ij}=0$. So our matrix $Q$ is
\[
  Q=\begin{bmatrix}
    \left(\frac{b}{b-a}\right)^2 & 0 & 0\\
    0 & r^2 & 0\\
    0 & 0 & 1
  \end{bmatrix}
\]

Now we need the unit vectors that point along each axis. They are $u_1=\widehat{q_1},u_2=\widehat{q_2},u_3=\widehat{q_3}$,
then
\[
  g^{-1}=
  [u_iu_j]=
  \begin{bmatrix}
    1 & 0 & 0\\
    0 & 1 & 0\\
    0 & 0 & 1
  \end{bmatrix}
\]
Then $\hat{\mu}_{ij}=0$ for $i\neq j$ and for $i=j$ we have
\begin{align*}
  \hat{\mu}_{11}=\mu|u_1\cdot(u_2\times u_3)|Q_1Q_2Q_3Q_1^{-2}
  &=\mu|u_1\cdot(u_2\times u_3)|Q_2Q_3Q_1^{-1}\\
  &=\mu|u_1\cdot(u_2\times u_3)|Q_1^{-1}\\
  &=\mu|u_1\cdot\hat{r}|Q_1^{-1}\\
  &=\mu Q_1^{-1}\qquad (???)\\
  &=\mu \frac{R_2-R_1}{R_2}
\end{align*}

\begin{align*}
  \hat{\mu}_{22}=\mu|u_1\cdot(u_2\times u_3)|Q_1Q_2Q_3Q_2^{-2}
  &=\mu|u_1\cdot(u_2\times u_3)|Q_1Q_3Q_2^{-1}\\
  &=\mu|u_1\cdot(u_2\times u_3)|Q_1\\
  &=\mu|u_1\cdot\hat{r}|Q_1\\
  &=\mu Q_1\qquad (????)\\
  &=\mu \frac{R_2}{R_2-R_1}
\end{align*}

\begin{align*}
  \hat{\mu}_{33}=\mu|u_1\cdot(u_2\times u_3)|Q_1Q_2Q_3Q_3^{-2}
  &=\mu|u_1\cdot(u_2\times u_3)|Q_1Q_2Q_3^{-1}\\
  &=\mu|u_1\cdot(u_2\times u_3)|Q_1\\
  &=\mu|u_1\cdot\hat{r}|Q_1\\
  &=\mu Q_1\qquad (????)\\
  &=\mu \frac{R_2}{R_2-R_1}
\end{align*}

% \printbibliography

\section{Example: cylindrical cloaking}
This example is a famous cloak proposed in \cite{schurig06}.

\end{document}
