\documentclass{article}

\usepackage{amsfonts}
\usepackage{amsmath}
\usepackage{biblatex}
\usepackage{graphicx}
\usepackage{xcolor}

\addbibresource{dgproject.bib}

\begin{document}

\newcommand{\R}{\mathbb{R}}

\title{MATH 670 Project Description}
\author{Fernando}
\date{\today}
\maketitle

\section*{Project title and format}

\textbf{Title:} An application of differential geometry to magnetic cloaking.\\
\textbf{Format:} Paper?

\section*{Project description}

The main idea in cloaking is to hide (cloak) a specific region of space from an
incident wave. The goal is that after the wave passes through the cloak it is
undistinguishable from the wave we would see if the object was not there,
making the region inside the cloak effectively invisible.

Cloaking often focuses on electromagnetic waves, but it is possible to cloak
mechanical waves too. For this project I would like to focus on magnetic
cloaking, i.e. the goal here is to take a magnet and put it inside our cloak,
then we should be able to move the cloak (with the magnet inside it) through a
magnetic field as if it were a non-magnetic material, i.e. the magnetic field
should not exert any force on the magnet (nor the cloak!) and vice versa.

There are different ways to approach this, one option is to use geometrical
transformations; this technique uses the fact that Maxwell's equations are form
invariant under change of coordinates. Te procedure is as follows:
\begin{enumerate}
\item

Make a geometric transformation that distorts the space in a way that avoids
the region we want to cloak.

\item

Apply this geometric transformation to Maxwell's equations.

\item

In this new coordinate system, modify Maxwell's equations to have the same
form they did in the original coordinates (this will change the constants of
the equation).

\item

The constants obtained in the previous step will tell us the material
properties we need for our cloak, then we can try to build it.

\end{enumerate}


With this application in mind what I would like to do for the project is to
understand and explain the differential geometry part, focusing on steps two 2
and 3.
\cite{ward96}

% \printbibliography

\end{document}
