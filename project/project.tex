\documentclass{article}
\usepackage{amsmath}
\usepackage{amsfonts}
\usepackage{graphicx}
\usepackage{xcolor}

\begin{document}

\newcommand{\R}{\mathbb{R}}

\title{Project Description MATH 670}
\author{Fernando}
\date{\today}
\maketitle

\section*{Project title}

An application of differential geometry to magnetic cloaking.

\section*{Project format}
Paper

\section*{Project description}

The main idea in cloaking is to hide (cloak) a
specific region of space from an incident wave. The goal is that after the wave
passes through the cloak it is undistinguishable from the wave we would had
seen if the object was not there, making the region inside the cloak
effectively invisible.

Cloaking often focuses on electromagnetic waves, but it is possible to cloak
mechanical waves too. For this project I would like to focus on magnetic
cloaking, i.e. the goal here is to take a magnet and put it inside our cloak,
then we should be able to move the cloak (with the magnet inside it) through a
magnetic field as if it were a non-magnetic material, i.e. the magnetic field
should not exert any force on the magnet (nor the cloak!) and vice versa.

\end{document}
