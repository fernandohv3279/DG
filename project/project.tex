% \documentclass{article}
\documentclass{amsart}

\usepackage{amsfonts}
\usepackage{amsmath}
\usepackage{biblatex}
\usepackage{blindtext}
\usepackage{mathtools}
\usepackage{xcolor}

\addbibresource{project.bib}

\begin{document}

\newcommand{\R}{\mathbb{R}}
\newcommand\tbf[1]{\textbf{#1}}
\newcommand\myworries[1]{\textcolor{red}{\tbf{#1}}}
% \renewcommand\myworries[1]{}

\title{An application of differential geometry to magnetic cloaking}
\author{Fernando}
\date{\today}
\maketitle

\myworries{Improve titles}
\section{Discalimer}
I did absolutely nothing. We=they.
\myworries{How to write this?}
\section{Intro}
\blindtext[1]
\section{Theory}
This section was taken from \cite{ward96}, specifically sections 1 and 2.
\myworries{\\Reorganize this section??}
\subsection{Summary of section 1 from pendry's article}
We start with Maxewell's equations:
\begin{align*}
  \nabla \times \tbf{E} &= -\mu \mu_0 \partial \tbf{H} /\partial t\\
  \nabla \times \tbf{H} &= +\varepsilon \varepsilon_0 \partial \tbf{E} /\partial t,
\end{align*}
and then consider a transformation
\[
q_1(x,y,z), \quad q_2(x,y,z), \quad q_3(x,y,z).
\]

\tbf{Question:} What form does the new set of equations take when
expressed in terms of $q_1,q_2,q_3$?

\tbf{Answer:} In the new system of coordinates, Maxwell's equations become
\begin{align*}
  \nabla_q \times \widehat{\tbf{E}} &= -\mu_0\hat{\mu} \partial \widehat{\tbf{H}} \tbf{*}\partial t \myworries{ What is the asterisk notation?}\\
  \nabla_q \times \widehat{\tbf{H}} &= -\varepsilon_0 \hat{\varepsilon} \partial \widehat{\tbf{E}}\tbf{*}\partial t \myworries{ Why is there a - here?},
\end{align*}
where $\hat{\mu}$ and $\hat{\varepsilon}$ are in general tensors and $\hat{\tbf{E}},\hat{\tbf{H}}$ are renormalized electric and magnetic fields.\myworries{ What does this mean? Renormalized?}
In other words, the form of Maxewell's equations is preserved by a coordinate transformation, it just changes the definition of $\varepsilon,\mu$.
\subsection{Summary of section 2 from pendry's article}
We define three unit vectors $\tbf{u}_1,\tbf{u}_2,\tbf{u}_3$
\section{Geometric transformation 1}
Let's do the computation in \cite{ward96} with the following transformation:
\[\begin{cases}
r'=R_1+r(R_2-R_1)/R_2\\
\theta'=\theta\\
z'=z\\
\end{cases}\]
In terms of the original variables this is
\[\begin{cases}
r=\frac{r'-R_1}{R_2-R_1}R_2=\frac{R_2}{R_2-R_1}r'-\frac{R_1R_2}{R_2-R_1}\\
\theta=\theta'\\
z=z'\\
\end{cases}\]
Then following the formula from \cite{ward96} we have
\begin{align*}
Q_{11}
=&\frac{\partial r}{\partial r'}\frac{\partial r}{\partial r'}
+\frac{\partial \theta}{\partial r'}\frac{\partial \theta}{\partial r'}
+\frac{\partial z}{\partial r'}\frac{\partial z}{\partial r'}\\
=&\frac{\partial r}{\partial r'}\frac{\partial r}{\partial r'}\\
=&\left(\frac{R_2}{R_2-R_1}\right)^2
\end{align*}

\begin{align*}
Q_{22}
=&\frac{\partial r}{\partial \theta'}\frac{\partial r}{\partial \theta'}
+\frac{\partial \theta}{\partial \theta'}\frac{\partial \theta}{\partial \theta'}
+\frac{\partial z}{\partial \theta'}\frac{\partial z}{\partial \theta'}\\
=&\frac{\partial \theta}{\partial \theta'}\frac{\partial \theta}{\partial \theta'}\\
=&1
\end{align*}

\begin{align*}
Q_{33}
=&\frac{\partial r}{\partial z'}\frac{\partial r}{\partial z'}
+\frac{\partial \theta}{\partial z'}\frac{\partial \theta}{\partial z'}
+\frac{\partial z}{\partial z'}\frac{\partial z}{\partial z'}\\
=&\frac{\partial z}{\partial z'}\frac{\partial z}{\partial z'}\\
=&1
\end{align*}

For $Q_{ij}$ where $i\neq j$ we have $Q_{ij}=0$. So our matrix $Q$ is
\[
  Q=\begin{bmatrix}
    \left(\frac{R_2}{R_2-R_1}\right)^2 & 0 & 0\\
    0 & 1 & 0\\
    0 & 0 & 1
  \end{bmatrix}
\]

Now we need the unit vectors that point along each axis. They are $u_1=\hat{r},u_2=\hat{\theta},u_3=\hat{z}$.

then
\[
  g^{-1}=
  [u_iu_j]=
  \begin{bmatrix}
    \hat{r}\cdot\hat{r} & \hat{r}\cdot\hat{\theta} & \hat{r}\cdot\hat{z}\\
    \hat{\theta}\cdot\hat{r} & \hat{\theta}\cdot\hat{\theta} & \hat{\theta}\cdot\hat{z}\\
    \hat{z}\cdot\hat{r} & \hat{z}\cdot\hat{\theta} & \hat{z}\cdot\hat{z}
  \end{bmatrix}
  =
  \begin{bmatrix}
    1 & 0 & 0\\
    0 & 1 & 0\\
    0 & 0 & 1
  \end{bmatrix}
\]
Then $\hat{\mu}_{ij}=0$ for $i\neq j$ and for $i=j$ we have
\begin{align*}
  \hat{\mu}_{11}=\mu|u_1\cdot(u_2\times u_3)|Q_1Q_2Q_3Q_1^{-2}
  &=\mu|u_1\cdot(u_2\times u_3)|Q_2Q_3Q_1^{-1}\\
  &=\mu|u_1\cdot(u_2\times u_3)|Q_1^{-1}\\
  &=\mu|u_1\cdot\hat{r}|Q_1^{-1}\\
  &=\mu Q_1^{-1}\qquad (???)\\
  &=\mu \frac{R_2-R_1}{R_2}
\end{align*}

\begin{align*}
  \hat{\mu}_{22}=\mu|u_1\cdot(u_2\times u_3)|Q_1Q_2Q_3Q_2^{-2}
  &=\mu|u_1\cdot(u_2\times u_3)|Q_1Q_3Q_2^{-1}\\
  &=\mu|u_1\cdot(u_2\times u_3)|Q_1\\
  &=\mu|u_1\cdot\hat{r}|Q_1\\
  &=\mu Q_1\qquad (????)\\
  &=\mu \frac{R_2}{R_2-R_1}
\end{align*}

\begin{align*}
  \hat{\mu}_{33}=\mu|u_1\cdot(u_2\times u_3)|Q_1Q_2Q_3Q_3^{-2}
  &=\mu|u_1\cdot(u_2\times u_3)|Q_1Q_2Q_3^{-1}\\
  &=\mu|u_1\cdot(u_2\times u_3)|Q_1\\
  &=\mu|u_1\cdot\hat{r}|Q_1\\
  &=\mu Q_1\qquad (????)\\
  &=\mu \frac{R_2}{R_2-R_1}
\end{align*}

% \printbibliography

\end{document}
