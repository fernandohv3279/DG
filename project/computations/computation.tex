% \documentclass{article}
\documentclass{amsart}

\usepackage{amsfonts}
\usepackage{amsmath}
\usepackage{biblatex}
\usepackage{blindtext}
\usepackage{mathtools}
\usepackage{xcolor}

\addbibresource{project.bib}

\begin{document}

\newcommand{\R}{\mathbb{R}}
\newcommand\tbf[1]{\textbf{#1}}
\newcommand\myworries[1]{\textcolor{red}{\tbf{#1}}}
% \renewcommand\myworries[1]{}

\title{Computations}
\date{\today}
\maketitle

\section{Desired result}

According to \cite{smith06} the transformation
\[\begin{cases}
  r'=a+r(b-a)/b\\
  \theta'=\theta\\
  z'=z\\
\end{cases}\]
gives the following parameters
\[\begin{cases}
  \varepsilon_r=\mu_r=\frac{r-a}{r}\\
  \varepsilon_\theta=\mu_\theta=\frac{r}{r-a}\\
  \varepsilon_z=\mu_z=\left(\frac{b}{b-a}\right)^2\frac{r-a}{r}\\
\end{cases}\]
I would like to understand this computation.

\section{Computing the parameters}
This is explained in \cite{ward96}, where they present formulas for the new parameters.
Such formulas are (here we assume that permeability and permitivity are both 1):
\begin{equation}
  \hat{\mu}^{ij}=\hat{\varepsilon}^{ij}=g^{ij}|u_1\cdot(u_2\times u_3)|Q_1Q_2Q_3Q_3(Q_iQ_j)^{-1}.
\end{equation}
Now let's figure out each of the terms in this expression.

To obtain the parameters we need to the following quantities:

\begin{equation}\label{matrixQ}
  Q_{ij}=\frac{\partial x}{\partial q_i}\frac{\partial x}{\partial q_j}+\frac{\partial y}{\partial q_i}\frac{\partial y}{\partial q_j}+\frac{\partial z}{\partial q_i}\frac{\partial z}{\partial q_j},
\end{equation}
and for convenience they denote $Q_i=\sqrt{Q_{ii}}$.

We define three unit vectors $\tbf{u}_1,\tbf{u}_2,\tbf{u}_3,$ to point along the generalized $q_1,q_2,q_3$ axes.
We also define

Another definition we need is
\begin{equation}\label{matrixg}
  g^{-1}=[u_i\cdot u_j]=\begin{bmatrix}
    \tbf{u}_1\cdot \tbf{u}_1 & \tbf{u}_1\cdot \tbf{u}_2 & \tbf{u}_1\cdot \tbf{u}_3 \\
    \tbf{u}_2\cdot \tbf{u}_1 & \tbf{u}_2\cdot \tbf{u}_2 & \tbf{u}_2\cdot \tbf{u}_3 \\
    \tbf{u}_3\cdot \tbf{u}_1 & \tbf{u}_3\cdot \tbf{u}_2 & \tbf{u}_3\cdot \tbf{u}_3
  \end{bmatrix}.
\end{equation}
Now we are ready to define $\hat{\mu},\hat{\varepsilon}$. According to \cite{ward96} they are
\begin{align}
  \hat{\mu}^{ij}=\mu g^{ij}|\tbf{u}_1\cdot(\tbf{u}_2\times \tbf{u}_3)|Q_1Q_2Q_3(Q_iQ_j)^{-1}\\
  \hat{\varepsilon}^{ij}=\varepsilon g^{ij}|\tbf{u}_1\cdot(\tbf{u}_2\times \tbf{u}_3)|Q_1Q_2Q_3(Q_iQ_j)^{-1} \nonumber
\end{align}

\section{Geometric transformation 1}
The transformation we want is the following
\[\begin{cases}
  q_1(r,\theta,z)=a+r(b-a)/b\\
  q_2(r,\theta,z)=\theta\\
  q_3(r,\theta,z)=z\\
\end{cases}\]
Notice that it is writen in terms of cylindrical coordinates
(from cylindrical to cylindrical), but we can move from cylindrical
to cartesian with the usual change of variables:
\begin{equation}
  \begin{cases}\label{cylindrical}
  x=r\cos\theta\\
  y=r\sin\theta\\
  z=z\\
\end{cases}
\end{equation}

Let's apply the formulas from the previous section to the transformation:
\[\begin{cases}
  q_1(x,y,z)=a+r(b-a)/b\\
  q_2(x,y,z)=\theta\\
q_3(x,y,z)=z\\
\end{cases}.\]
Here I kept the transformation in terms of $r,\theta,z$ in order to have cleaner looking formulas.
We can write $r,\theta,z$ in terms of $x,y,z$ if we wanted.

Now we neeq $x,y,z$ in terms of $q_1,q_2,q_3$, for this we note that:
\[\begin{cases}
r=\frac{q_1-a}{b-a}b=\frac{b}{b-a}q_1-\frac{ba}{b-a}\\
\theta=q_2\\
z=q_3\\
\end{cases}.\]
Again this is in terms of $r,\theta,z$ for simplicity, but combining this with (\ref{cylindrical})
we get $x,y,z$ in terms of $q_1,q_2,q_3$.

Now we use (\ref{matrixQ}) to compute $Q_{ij}$.
\begin{align*}
Q_{11}
=&\frac{\partial x}{\partial q_1}\frac{\partial x}{\partial q_1}
+\frac{\partial y}{\partial q_1}\frac{\partial y}{\partial q_1}
+\frac{\partial z}{\partial q_1}\frac{\partial z}{\partial q_1}\\
=&\left(\frac{\partial x}{\partial q_1}\right)^2+\left(\frac{\partial y}{\partial q_1}\right)^2+0\\
=&\left(\frac{\partial (r\cos\theta)}{\partial q_1}\right)^2+\left(\frac{\partial (r\sin\theta)}{\partial q_1}\right)^2\\
=&\left(\cos\theta\frac{\partial r}{\partial q_1}\right)^2+\left(\sin\theta\frac{\partial r}{\partial q_1}\right)^2\\
=&\left(\frac{\partial r}{\partial q_1}\right)^2\\
=&\left(\frac{b}{b-a}\right)^2
\end{align*}

\begin{align*}
Q_{22}
=&\frac{\partial x}{\partial q_2}\frac{\partial x}{\partial q_2}
+\frac{\partial y}{\partial q_2}\frac{\partial y}{\partial q_2}
+\frac{\partial z}{\partial q_2}\frac{\partial z}{\partial q_2}\\
=&\left(\frac{\partial x}{\partial q_2}\right)^2+\left(\frac{\partial y}{\partial q_2}\right)^2+0\\
=&\left(r\frac{\partial \cos\theta}{\partial q_2}\right)^2+\left(r\frac{\partial \sin\theta}{\partial q_2}\right)^2\\
=&\left(-r\sin\theta\frac{\partial \theta}{\partial q_2}\right)^2+\left(r\cos\theta\frac{\partial \theta}{\partial q_2}\right)^2\\
=&r^2\left(\frac{\partial \theta}{\partial q_2}\right)^2\\
=&r^2
\end{align*}

\begin{align*}
Q_{33}
=&\frac{\partial r}{\partial q_3}\frac{\partial r}{\partial q_3}
+\frac{\partial \theta}{\partial q_3}\frac{\partial \theta}{\partial q_3}
+\frac{\partial z}{\partial q_3}\frac{\partial z}{\partial q_3}\\
=&0+0+1\\
=&1
\end{align*}

For $Q_{ij}$ where $i\neq j$ we have $Q_{ij}=0$. So our matrix $Q$ is
\[
  Q=\begin{bmatrix}
    \left(\frac{b}{b-a}\right)^2 & 0 & 0\\
    0 & r^2 & 0\\
    0 & 0 & 1
  \end{bmatrix}
\]

Now we need the unit vectors that point along each axis. They are $u_1=\widehat{q_1},u_2=\widehat{q_2},u_3=\widehat{q_3}$,
then
\[
  g^{-1}=
  [u_iu_j]=
  \begin{bmatrix}
    1 & 0 & 0\\
    0 & 1 & 0\\
    0 & 0 & 1
  \end{bmatrix}
\]
Then $\hat{\mu}_{ij}=0$ for $i\neq j$ and for $i=j$ we have
\begin{align*}
  \hat{\mu}_{11}=\mu|u_1\cdot(u_2\times u_3)|Q_1Q_2Q_3Q_1^{-2}
  &=\mu|u_1\cdot(u_2\times u_3)|Q_2Q_3Q_1^{-1}\\
  &=\mu|u_1\cdot(u_2\times u_3)|Q_1^{-1}\\
  &=\mu|u_1\cdot\hat{r}|Q_1^{-1}\\
  &=\mu Q_1^{-1}\qquad (???)\\
  &=\mu \frac{R_2-R_1}{R_2}
\end{align*}

\begin{align*}
  \hat{\mu}_{22}=\mu|u_1\cdot(u_2\times u_3)|Q_1Q_2Q_3Q_2^{-2}
  &=\mu|u_1\cdot(u_2\times u_3)|Q_1Q_3Q_2^{-1}\\
  &=\mu|u_1\cdot(u_2\times u_3)|Q_1\\
  &=\mu|u_1\cdot\hat{r}|Q_1\\
  &=\mu Q_1\qquad (????)\\
  &=\mu \frac{R_2}{R_2-R_1}
\end{align*}

\begin{align*}
  \hat{\mu}_{33}=\mu|u_1\cdot(u_2\times u_3)|Q_1Q_2Q_3Q_3^{-2}
  &=\mu|u_1\cdot(u_2\times u_3)|Q_1Q_2Q_3^{-1}\\
  &=\mu|u_1\cdot(u_2\times u_3)|Q_1\\
  &=\mu|u_1\cdot\hat{r}|Q_1\\
  &=\mu Q_1\qquad (????)\\
  &=\mu \frac{R_2}{R_2-R_1}
\end{align*}

% \printbibliography

\end{document}
