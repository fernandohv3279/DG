% \documentclass{article}
\documentclass{amsart}

\usepackage{amsfonts}
\usepackage{amsmath}
\usepackage{biblatex}
\usepackage{blindtext}
\usepackage{mathtools}
\usepackage{xcolor}

\addbibresource{project.bib}

\begin{document}

\newcommand{\R}{\mathbb{R}}
\newcommand\tbf[1]{\textbf{#1}}
\newcommand\myworries[1]{\textcolor{red}{\tbf{#1}}}
% \renewcommand\myworries[1]{}

\title{Computations}
\date{\today}
\maketitle

\section{Desired result}

According to \cite{smith06} the transformation
\begin{equation}\label{THEtransCyl}
  \begin{cases}
  r'=a+r(b-a)/b\\
  \theta'=\theta\\
  z'=z\\
\end{cases}\end{equation}
gives the following parameters
\begin{equation}\begin{cases}
  \varepsilon_r=\mu_r=\frac{r-a}{r}\\
  \varepsilon_\theta=\mu_\theta=\frac{r}{r-a}\\
  \varepsilon_z=\mu_z=\left(\frac{b}{b-a}\right)^2\frac{r-a}{r}\\
\end{cases}\end{equation}
I would like to understand this computation.

\section{Computing the parameters}
This is explained in \cite{ward96}, where they present formulas for the new parameters.
Such formulas are (here we assume that permeability and permitivity are both 1):
\begin{equation}
  \hat{\mu}^{ij}=\hat{\varepsilon}^{ij}=g^{ij}|u_1\cdot(u_2\times u_3)|Q_1Q_2Q_3Q_3(Q_iQ_j)^{-1}.
\end{equation}
Now let's figure out each of the terms in this expression.

\subsection{Computing $Q_i$} If we change to a coordinate system
\[q_1(x,y,z),q_2(x,y,z),q_3(x,y,z)\]
then \cite{ward96} defines
\begin{equation}\label{matrixQ}
  Q_{ij}=\frac{\partial x}{\partial q_i}\frac{\partial x}{\partial q_j}+\frac{\partial y}{\partial q_i}\frac{\partial y}{\partial q_j}+\frac{\partial z}{\partial q_i}\frac{\partial z}{\partial q_j},
\end{equation}
and then $Q_i=\sqrt{Q_{ii}}$.

Let's compute these quantities for transformation (\ref{THEtransCyl}).

The first observation is that the transformation is written in terms of cylindrical coordinates,
but since we know that $r=\sqrt{x^2+y^2},\theta=\tan^{-1}(y/x)$, we can write
\begin{equation}\label{THEtransCart}
  \begin{cases}
  q_1(x,y,z)=a+r(x,y)(b-a)/b\\
  q_2(x,y,z)=\theta(x,y)\\
  q_3(x,y,z)=z\\
\end{cases}\end{equation}
Inverting this we get:
\[\begin{cases}
  r=(q_1-a)b/(b-a)\\
  \theta=q_2\\
  z=q_3\\
\end{cases}\]
And since $x=r\cos\theta,y=r\sin\theta$ then we can perform the following computations
\begin{itemize}
  \item $\frac{\partial x}{\partial q_1}=\frac{\partial (r\cos\theta)}{\partial q_1}=\cos\theta\frac{\partial r}{\partial q_1}=\frac{b}{b-a}\cos\theta$
  \item $\frac{\partial y}{\partial q_1}=\frac{\partial (r\sin\theta)}{\partial q_1}=\sin\theta\frac{\partial r}{\partial q_1}=\frac{b}{b-a}\sin\theta$
  \item $\frac{\partial z}{\partial q_1}=0$
  \item $\frac{\partial x}{\partial q_2}=\frac{\partial (r\cos\theta)}{\partial q_2}=r\frac{\partial \cos\theta}{\partial q_2}=-r\sin\theta$
  \item $\frac{\partial y}{\partial q_2}=\frac{\partial (r\sin\theta)}{\partial q_2}=r\frac{\partial \sin\theta}{\partial q_2}=r\cos\theta$
  \item $\frac{\partial z}{\partial q_2}=0$
  \item $\frac{\partial x}{\partial q_3}=0$
  \item $\frac{\partial y}{\partial q_3}=0$
  \item $\frac{\partial z}{\partial q_3}=1$
\end{itemize}
Now we can compute $Q_{ij}$ and we get:
\begin{itemize}
  \item $Q_{11}=\left(\frac{b}{b-a}\right)^2$
  \item $Q_{12}=Q_{21}=0$
  \item $Q_{13}=Q_{31}=0$
  \item $Q_{22}=r^2$
  \item $Q_{23}=Q_{32}=0$
  \item $Q_{33}=1$
\end{itemize}
So the matrix we obtain is:
\[
  Q=\begin{bmatrix}
    \left(\frac{b}{b-a}\right)^2 & 0 & 0\\
    0 & r^2 & 0\\
    0 & 0 & 1
  \end{bmatrix}.
\]

\subsection{Computing $u_i$}

In \cite{ward96} $u_1,u_2,u_3$ are defined as unit vectors that point along the generalized $q_1,q_2,q_3$ axes.

We compute them using (\ref{THEtransCart}):\myworries{ I suspect this is where the mistake is}
\begin{align*}
\begin{bmatrix}
  |  & |  & |\\
  u_1 & u_2 & u_3\\
  |  & |  & |\\
\end{bmatrix}
&=
\begin{bmatrix}
  \frac{\partial q_1}{\partial x} & \frac{\partial q_1}{\partial y} & \frac{\partial q_1}{\partial z}\\
  \frac{\partial q_2}{\partial x} & \frac{\partial q_2}{\partial y} & \frac{\partial q_2}{\partial z}\\
  \frac{\partial q_3}{\partial x} & \frac{\partial q_3}{\partial y} & \frac{\partial q_3}{\partial z}
\end{bmatrix}\\
&=
\begin{bmatrix}
  \frac{b-a}{b}\frac{\partial r}{\partial x} & \frac{b-a}{b}\frac{\partial r}{\partial y} & 0\\
  \frac{\partial q_2}{\partial x}& \frac{\partial q_2}{\partial y} & 0\\
  0 & 0 & 1
\end{bmatrix}\\
&=
\begin{bmatrix}
  \frac{b-a}{b}\frac{x}{\sqrt{x^2+y^2}} & \frac{b-a}{b}\frac{y}{\sqrt{x^2+y^2}} & 0\\
  \frac{1}{1+(y/x)^2}\cdot\frac{-y}{x^2}& \frac{1}{1+(y/x)^2}\cdot\frac{1}{x} & 0\\
  0 & 0 & 1
\end{bmatrix}\\
&=
\begin{bmatrix}
  \frac{b-a}{b}\frac{x}{\sqrt{x^2+y^2}} & \frac{b-a}{b}\frac{y}{\sqrt{x^2+y^2}} & 0\\
  \frac{-y}{x^2+y^2}& \frac{x}{x^2+y^2} & 0\\
  0 & 0 & 1
\end{bmatrix}\\
&=
\begin{bmatrix}
  \frac{b-a}{b}\cos\theta & \frac{b-a}{b}\sin\theta & 0\\
  -\frac{\sin\theta}{r}& \frac{\cos\theta}{r} & 0\\
  0 & 0 & 1
\end{bmatrix}
\end{align*}
Let's remember that the map we are working with is
\[\begin{cases}
  q_1(x,y,z)=a+r(x,y)(b-a)/b\\
  q_2(x,y,z)=\theta(x,y)\\
  q_3(x,y,z)=z\\
\end{cases}\]

Another definition we need is
\begin{equation}\label{matrixg}
  g^{-1}=[u_i\cdot u_j]=\begin{bmatrix}
    \tbf{u}_1\cdot \tbf{u}_1 & \tbf{u}_1\cdot \tbf{u}_2 & \tbf{u}_1\cdot \tbf{u}_3 \\
    \tbf{u}_2\cdot \tbf{u}_1 & \tbf{u}_2\cdot \tbf{u}_2 & \tbf{u}_2\cdot \tbf{u}_3 \\
    \tbf{u}_3\cdot \tbf{u}_1 & \tbf{u}_3\cdot \tbf{u}_2 & \tbf{u}_3\cdot \tbf{u}_3
  \end{bmatrix}.
\end{equation}

Now we need the unit vectors that point along each axis. They are $u_1=\widehat{q_1},u_2=\widehat{q_2},u_3=\widehat{q_3}$,
then
\[
  g^{-1}=
  [u_iu_j]=
  \begin{bmatrix}
    1 & 0 & 0\\
    0 & 1 & 0\\
    0 & 0 & 1
  \end{bmatrix}
\]
Then $\hat{\mu}_{ij}=0$ for $i\neq j$ and for $i=j$ we have
\begin{align*}
  \hat{\mu}_{11}=\mu|u_1\cdot(u_2\times u_3)|Q_1Q_2Q_3Q_1^{-2}
  &=\mu|u_1\cdot(u_2\times u_3)|Q_2Q_3Q_1^{-1}\\
  &=\mu|u_1\cdot(u_2\times u_3)|Q_1^{-1}\\
  &=\mu|u_1\cdot\hat{r}|Q_1^{-1}\\
  &=\mu Q_1^{-1}\qquad (???)\\
  &=\mu \frac{R_2-R_1}{R_2}
\end{align*}

\begin{align*}
  \hat{\mu}_{22}=\mu|u_1\cdot(u_2\times u_3)|Q_1Q_2Q_3Q_2^{-2}
  &=\mu|u_1\cdot(u_2\times u_3)|Q_1Q_3Q_2^{-1}\\
  &=\mu|u_1\cdot(u_2\times u_3)|Q_1\\
  &=\mu|u_1\cdot\hat{r}|Q_1\\
  &=\mu Q_1\qquad (????)\\
  &=\mu \frac{R_2}{R_2-R_1}
\end{align*}

\begin{align*}
  \hat{\mu}_{33}=\mu|u_1\cdot(u_2\times u_3)|Q_1Q_2Q_3Q_3^{-2}
  &=\mu|u_1\cdot(u_2\times u_3)|Q_1Q_2Q_3^{-1}\\
  &=\mu|u_1\cdot(u_2\times u_3)|Q_1\\
  &=\mu|u_1\cdot\hat{r}|Q_1\\
  &=\mu Q_1\qquad (????)\\
  &=\mu \frac{R_2}{R_2-R_1}
\end{align*}

% \printbibliography

\end{document}
